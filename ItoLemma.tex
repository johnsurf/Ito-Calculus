%   11/11/92 211121724  MEMBER NAME  REPORT1  (TEXFILES) M  TEX
%format: latex    %hyphen: English
%\documentstyle[11pt,leqno,epsfig]{note}
%\documentstyle[11pt,epsfig]{note}
%\documentstyle[11pt,epsfig,pstimes,psmath]{note}
%\documentstyle[epsf]{article}
 
%\documentclass{article}
\documentclass[epsf]{article}
\usepackage{graphicx}

\font\twelverm=cmr12
\font\twelvebf=cmbx12
\font\tenbf=cmbx10
\font\tenrm=cmr10
\font\tenit=cmti10
\font\elevenbf=cmbx10 scaled\magstep 1
\font\elevenrm=cmr10 scaled\magstep 1
\font\elevenit=cmti10 scaled\magstep 1
\font\ninebf=cmbx9
\font\ninerm=cmr9
\font\nineit=cmti9
\font\eightbf=cmbx8
\font\eightrm=cmr8
\font\eightit=cmti8
\font\sevenrm=cmr7

%\usepackage{amsmath,amssymb,latexsym} 
\usepackage{amsmath,amssymb,latexsym,hyperref} 
% \hypersetup{
%  colorlinks=true,
%    linkcolor=blue,\vspace{.2in}
%    \centerline {
%    \includegraphics[width=2in]{f1}
%    }
%    \vspace{.2in}    
%    filecolor=magenta,      
%    urlcolor=cyan,
%}

\hypersetup{
    bookmarks=true,          % show bookmarks bar?
    unicode=false,             % non-Latin characters in Acrobat�s bookmarks
    pdftoolbar=true,           % show Acrobat�s toolbar?
    pdfmenubar=true,        % show Acrobat�s menu?
    pdffitwindow=false,      % window fit to page when opened
    pdfstartview={FitH},     % fits the width of the page to the window
    pdftitle={My title},         % title
    pdfauthor={Author},     % author
    pdfsubject={Subject},   % subject of the document
    pdfcreator={Creator},   % creator of the document
    pdfproducer={Producer}, % producer of the document
    pdfkeywords={keyword1, key2, key3}, % list of keywords
    pdfnewwindow=true,      % links in new PDF window
    colorlinks=true,               % false: boxed links; true: colored links
    linkcolor=red,                  % color of internal links (change box color with linkbordercolor)
    citecolor=green,             % color of links to bibliography
    filecolor=magenta,         % color of file links
    urlcolor=cyan                % color of external links
}

%
%  `pstimes.sty' sets \rm to be Times, and `psmath.sty' sets the math
%  in Times as well.
%
\textwidth = 7.0truein
%\textheight = 8.5truein
\oddsidemargin=-0.2truein
\evensidemargin=-0.2truein
\topmargin=0.0truein
%\notename{UCD/IIRPA} \notenumber{92-24}

\begin{document}
%\null
%\vspace{-2cm}
%\hbox to \hsize{\hfill UCD/IIRPA~92-24}
\begin{center}{{\twelvebf Notes on the Wiener Process and Ito's Lemma}}
		\vglue 1.0cm
 	       {\tenrm JOHN R.\,\,SMITH\\}
	       {email: jrsmith@ucdavis.edu\\}
		\vglue 0.25cm
	        \baselineskip=13pt
	       {\tenit Physics Department\\
	        University of California, Davis\\}
                 \baselineskip=12pt
                 {\tenit Davis, CA 95616-8677, USA\\}
		\vglue 0.5cm
                 {\rm August 23, 2017 \\}
\end{center}

%
%\newmathalphabet{\oldcal}
%\addtoversion{times}{\oldcal}{cmsy}{m}{n}
\newcommand{\unit}[1]{\mbox{\elevenit #1}}
\newcommand{\subs}[1]{\mbox{\scriptsize\elevenit #1}}
\def\Q2{$Q^2$}
\def\sgp{$\sigma_{\gamma p}\:$}
\mathchardef\Lcur="324C
\def\crfive{\cr\noalign{\vskip 5pt}}
\def\crten{\cr\noalign{\vskip 10pt}}
\def\cofac{\hbox{cofactor}}
\def\gij{g_{ij}}
\def\gijinv{g^{ij}}
\def\half{1\over2}
\def\gab{g_{\alpha\beta}}
\def\gabinv{g^{\alpha\beta}}
\def\gmunu{g_{\mu\nu}}
\def\gmunuinv{g^{\mu\nu}}
\def\gmu{\gamma^\mu}
\def\gmudag{\gamma^{\mu\dag}}
\def\gnu{\gamma^\nu}
\def\gmuc{\gamma_\mu}
\def\gnuc{\gamma_\nu}
\def\gzero{\gamma^0}
\def\gone{\gamma^1}
\def\gtwo{\gamma^2}
\def\gthree{\gamma^3}
\def\gfive{\gamma_5}
\def\gfive{\gamma_5}
\def\gzeroc{\gamma_0}
\def\gonec{\gamma_1}
\def\gtwoc{\gamma_2}
\def\gthreec{\gamma_3}
\def\qsq{{Q^2\over4E_1^2}}
\def\qr2{{Q^2\over2}}
\def\q2{{Q^2}}
\def\hy{\hat{y}}
\def\be{\begin{equation}}
\def\ee{\end{equation}}

\def\bearray{\begin{eqnarray*}}
\def\eearray{\end{eqnarray*}}
\def\Var{\hbox{Var}}

\def\mslim{\qopname\relax m{l.i.m}}

\newcommand{\bra}[1]{\langle #1|}
\newcommand{\ket}[1]{|#1\rangle}
\newcommand{\braket}[2]{\langle #1|#2\rangle}
\newcommand{\Tr}{{\rm Tr}}
\newcommand{\goes}{\rightarrow}
\newcommand{\define}{\overset{\Delta}{=} }
\newcommand{\ninfty}{n\rightarrow\infty}
\newcommand{\Rwp}{$\overset{\hbox{R}_{wp1}}{\sim}$}
\newcommand{\Rms}{$\overset{\hbox{R}_{ms}}{\sim}$}

\newtheorem{theorem}{Theorem}[section]
\newtheorem{lemma}[theorem]{Lemma}
\newtheorem{proposition}[theorem]{Proposition}
\newtheorem{corollary}[theorem]{Corollary}

\newenvironment{proof}[1][Proof]{\begin{trivlist}
\item[\hskip \labelsep {\bfseries #1}]}{\end{trivlist}}
\newenvironment{definition}[1][Definition]{\begin{trivlist}
\item[\hskip \labelsep {\bfseries #1}]}{\end{trivlist}}
\newenvironment{example}[1][Example]{\begin{trivlist}
\item[\hskip \labelsep {\bfseries #1}]}{\end{trivlist}}
\newenvironment{remark}[1][Remark]{\begin{trivlist}
\item[\hskip \labelsep {\bfseries #1}]}{\end{trivlist}}

\newcommand{\qed}{\nobreak \ifvmode \relax \else
      \ifdim\lastskip<1.5em \hskip-\lastskip
      \hskip1.5em plus0em minus0.5em \fi \nobreak
      \vrule height0.65em width0.4em depth0.20em\fi}
      %\vrule height0.75em width0.5em depth0.25em\fi}
      
\newcommand\xqed[1]{%
  \leavevmode\unskip\penalty9999 \hbox{}\nobreak\hfill
  \quad\hbox{#1}}
\newcommand\demo{\xqed{$\triangle$}}      
 


%\maketitle

%some definitions
%\setlength{\oddsidemargin}{0cm}
%\setlength{\textwidth}{17cm}
%\setlength{\topmargin}{-3cm}
%\setlength{\textheight}{25cm}
%\setlength{\unitlength}{1mm}
%\hsize=6.0 true in
%\vsize=8.40 true in
%\pagestyle{plain}

\parindent = 0.cm
%\vglue 0.3cm
%\rightskip=3pc
%\leftskip=3pc
\twelverm
\baselineskip=14pt
\noindent

%\input /Users/johnrsmith/tex/inputs/phyzzx.tex
%\input /Users/johnrsmith/tex/inputs/myphyx.tex
%$$p(x,D)\psi(x) = {1\over(2\pi)^n}\int_{R^n}\int_{K^n} e^{i k\cdot x}p(x,k)e^{-i k\cdot y}\,d^nk\psi(y)\,d^ny,\eqno(8)$$
%\begin{equation}p(x,D)\psi(x) = {1\over(2\pi)^n}\int_{R^n}\int_{K^n} e^{i k\cdot x}p(x,k)e^{-i k\cdot y}\,d^nk\psi(y)\,d^ny,\label{eqn:symbol}\end{equation}
%Eq.[\ref{eqn:symbol}]
%\begin{equation*}p(x,D)\psi(x) = {1\over(2\pi)^n}\int_{R^n}\int_{K^n} e^{i k\cdot x}p(x,k)e^{-i k\cdot y}\,d^nk\psi(y)\,d^ny,\end{equation*}

\input Introduction.tex
\input Probability.tex
\input SequencesOfSets.tex
\input RandomVariables.tex
\input Expectation.tex
\input Conditioning.tex
\section{Stochastic Processes}
\label{sec:stochastic}
 A stochastic process $\{X(t), t\in T\}$ is a parameterized collection of random variables. For every value of the parameter $t \in T, X(t)$ is a random variable. Often $t$ is taken to be time and $X(t)$ is referred to as the {\elevenit state}\/ of the stochastic process at time $t$. $T$ is the {\elevenit index set}\/ of the process and can consist of discrete values or can be in interval of the real line. When $T$ is a countable set the stochastic process is said to be a {\elevenit discrete-time process}\/ or a {\elevenit discrete parameter process} and denoted by $\{ X_t\}_{t\in T}$. If $T$ is an interval of the real line, then the stochastic process is said to be a {\elevenit continuous-time process}\/ or a {\elevenit continuous parameter process} and denoted, for example by $\{X(t), t\ge 0\}$ where $T$ is the halfline $[0,\infty)$. The {\elevenit state space}\/ of the process is defined to be the set of all possible values that the random variables $X(t)$ can take on. The random variables $X(t)$ can be scalars or vectors. If the values of $X(t)$ are discrete, then the stochastic process is said to have a {\elevenit discrete state space}. If they are continuous, the process is said to have a {\elevenit continuous state space}.\\

 The random variables that make up $X(t)$ are 
 defined on a probability space $(\Omega, {\cal F}, P)$ and assume values in $R$ if the process is scalar-valued or $\mathbb{R}^n$ if the process is vector-valued. A continuous parameter space $T$ is usually the halfline $[0,\infty)$ but it may also be an interval $[a,b]$, the non-negative integers in the case of a discrete parameter process or even subsets of $\mathbb{R}^n$ for $n\ge1$.\\

Using the notations $X_t(\omega)$ or $X(t,\omega)$) to represent the dependence of the stochastic process on $t$ as well as the random variable $\omega$, then for each $t\in T$ fixed we have a random variable $$\omega \rightarrow X_t(\omega), \quad \omega\in \Omega.$$ On the other hand, fixing $\omega \in \Omega$ we can consider the function $$t\rightarrow X_t(\omega), \quad t\in T$$ which is called the path of $X_t$. Once $\omega$ is selected, the entire path $X_t(\omega)$ is determined. \\

One can also consider the stochastic process as a function of two variables $t$ and $\omega$, i.e., $X(t,\omega)$ in place of $X_t(\omega)$. Then we can consider the stochastic process corresponding to the mapping $$(t,\omega) \rightarrow X_t(\omega)$$ from $T \times \Omega\rightarrow \mathbb{R}^n$.\\

The concepts of mean and variance given above for a statistical sample, can be extended accordingly into functions for stochastic processes as follows \\ \\
Mean: The mean function $\mu(t)$ is defined by  $$\mu(t) = E[X(t)]$$
Variance: The variance function $\sigma^2(t)$ is defined by $$\sigma^2(t) = \hbox{Var}[X(t)]$$
Autocorrelation: The autocorrelation function is given by $$\Gamma_X(t_1, t_2) = E[X(t_1) X(t_2)]$$
Autocovariance: The variance function alone is not enough to specify the second
             moments of a sequence of random variables. We therefore include the autocovariance function (covariance kernel) $$C_X(t_1, t_2) = E[ (X(t_1) - \mu(t_1) ) ( X(t_2) - \mu(t_2)) ]$$
             
We will confine ourselves to stochastic processes with $$E[X^2(t) ] < \infty$$for all t. Such stochastic processes are said to have {\elevenit finite average power}\/ and are also known as {\elevenit second-order processes}. The second moment of such a stochastic process is finite. This also implies by the Cauchy-Schwarz inequality that $E[|X(t)|]$ is finite (see Section [\ref{sec:CauchySchwarz}]. One interesting conclusion about process with finite $|X(t)|$ is $$P( |X| < \infty) = 1.$$ For if there were any finite non-zero probability that $|X| = \infty$, then $E[|X|] = \infty$. Therefore $P(|X| = \infty) = 0$.\\ 
 
Dropping the explicit dependence on $\omega$ (and using the notation $\{X(t)\}$) a stochastic process $X(t)$ for values of t given in steps as $$0\le t_0 < t_1 < ...< t_k$$ can be displayed by values $X({t_0}), X({t_1}), ..., X({t_k})$. A stochastic process $\{ X(t), t\in T\}$, whose index set $T$ is linear, is said to be \\

(1) {\elevenit strictly stationary of order k}, where $k$ is a positive integer, if for any points $t_1, \hdots, t_k \in T$ and any $h$ in $T$, the $k$-dimensional vectors
$$(X(t_1), \hdots, X(t_k)) \quad\hbox{and}\quad (X(t_1 + h), \hdots, X(t_k + h))$$are identically distributed.  \\

(2) {\elevenit strictly stationary}\/ if for any integer $k$ it is strictly stationary of order $k$.\\

A continuous parameter stochastic process $\{ X(t), 0\le t >\infty \}$ is said to have {\elevenit independent increments} if $X(0) = 0$ and, for all choices of the indices $t_0 < t_1 < \hdots t_k$, the $k$ 
random variables
$$X({t_1}) - X(t_0), X({t_2}) - X({t_1}), ..., X({t_k})- X({t_{k-1})}$$are independent. If in addition to being {\elevenit independent}, the increments have the property of {\elevenit stationarity}, or that $X({t+h})- X({s+h})$ has the same distribution as $X(t) - X(s)$ for any $t$ and $s$ and for all $h$, then the stochastic process is said to have  {\elevenit stationary independent increments}\, (SII). \\

Another notion of stationarity is {\elevenit covariance stationarity}: A stochastic process $\{X(t), t \in T\}$ is said to be covariance stationary if it has finite second moments, if its index set $T$ is linear, and if its covariance kernel $C_X(s,t)$ is a function only of the absolute difference $|s-t|$, in the sense that there exists a function $R(v)$ such that for all $s$ and $t$ in $T$ 

$$C_X(s,t) = R(s-t).$$ $R(v)$ has the property that for every $t$  and $v$ in $T$
$$\hbox{Cov}[X(t), X(t+v)] = R(v).$$

\section{Random Walk in One Dimension}
\label{sec:RandomWalks}
Let $U=(U_1, U_2, \hdots)$ be a sequence of independent random variables, each taking the values $-1$ and $1$ with probabilities $p\in[0,1]$ and $q = 1-p$ respectively. 
Let $V = (V_0, V_1, V_2  \hdots)$ be the partial sum process associated with $U$ starting with the initial step counter value $V_0 = 0$: 
$$V_0 = 0, \quad  V_n = V_0 + \sum_{i=1}^n U_i, \quad n = 1, 2, \hdots$$ The sequence $V_0, V_1, V_2 \hdots$ is called a simple random walk in one dimension with parameter $p$.\\

We imagine a person walking or a particle moving on an axis, so that at each discrete time step of duration $\tau$, the walker moves either one unit to the right 
with probability $p$ or one unit to the left with probability $1 - p$, independently from step to step. $V_n$ keeps count of the net number of steps in the positive direction from the origin.\\

The indicator function $I_k$ for $U_k$ is defined as follows: 
$$ I_k = \Big\{
\begin{array}{cl}
1 & \hbox{~if $U_k = ~~1$} \\ 
0 & \hbox{~if $U_k = -1$}
\end{array}
$$
In terms of $U_k$ we can write the indicator function as 
\be I_k = {1\over2}(U_k + 1). \label{eqn:Indicator} \ee The expectation of the Indicator Function satisfies: $$\displaystyle{E[I_k] = \sum_{x_k \in  \{I_k = 1\}} P(x_k)  = P\{I_k = 1\} = p}.$$  Also,  $E[I_k^2] = p$ by similar reasoning. Hence $$\hbox{Var}[I_k] = E[I_k^2] = (E[I_k])^2 = p - p^2 = p(1-p) = pq.$$
Using Eq. [\ref{eqn:Indicator}] we find that $U_k = 2I_k - 1$ and $U_k^2 = 4I_k^2 - 4I_k + 1$. Therefore the respective mean and variance of $U_k$ is given by 
\bearray  E[U_k] &=& 2p - 1 = p -q \\
 \hbox{Var}[U_k] &=& E[4I_k^2 - 4I_k + 1] - (2p-1)^2 \\
 &=& 1 - (2p-1)^2 = 1 -4p^2 + 4p - 1\\
 &=& 4p(1-p) = 4pq
\eearray
Since the $U_k, k = 1, 2, \hdots, I_n$ are a set of $n$ independent random variables, we find that 
\bearray
E[V_n] &=& n(p-q) \\
\hbox{Var}[V_n] &=& 4npq
\eearray
For large values of $n$ and moderate $p$ we can invoke the Central Limit Theorem (see Section \ref{sec:CentralLimit} and Theorem \ref{thm:CentralLimit}) with asymptotic mean and variance given by $\mu_n = n(p-q)$ and $\sigma_n^2 = 4npq$ respectively, to deduce the asymptotic limiting distribution of $V_n$ for large $n$ to be 

$$ P\Bigg\{ {V_n - \mu_n \over \sigma_n } \le a \Bigg\} \rightarrow {1\over \sqrt{2\pi}} \int_{-\infty}^a \exp\left(-{x^2 / 2}\right)\,dx.$$

Therefore the asymptotic probability density function for $V_n$ is given by the Gaussian: 

\be f_{V_n}(v) \approx {1\over\sqrt{8\pi npq}} \exp\Big\{- {(v - n[p-q])^2\over 8npq} \Big\} \label{eqn:netCounts}\ee 

The probability distribution for the random walker step count follows an approximate Gaussian probability distribution. Hence we anticipate that the same type of probability distribution will describe the position of a particle undergoing Brownian motion after $n$ ``steps'' (where each step is a collision that is assumed to knock the particle either to the right or to the left by the same distance $l$). If the length of each step in the random walk is $l$ then the position from the origin can be described in terms of $x = lv$.  Using Eq. [\ref{eqn:netCounts}] we can relate the probability that a particular undergoing the above ``random walk'' will be located in an interval $dx$ about $x$ is
\bearray
f_X(x) dx &=& f_{V_n}(v)\,dv = f_{V_n}(x/l) {dx\over l}\\
&\approx& {1\over\sqrt{2\pi} \sigma} \exp\Big\{- {(x - \mu)^2\over 2\sigma^2} \Big\}\, dx,%\label{eqn:netCounts}
\eearray where $\mu = nl(p-q)$ and $\sigma = 2l\sqrt{npq}$. From this derivation we see that the probability density for a random walker which starts at the origin to be 
within an interval of width $dx$ at $x$ after $n$ steps is given by 

 \be f_{X_n}(x) = {1\over\sqrt{2\pi} \sigma} \exp\Big\{- {(x- \mu)^2 / 2\sigma^2} \Big\} \label{eqn:walkGaussian}\ee 
Eq. [\ref{eqn:walkGaussian}] is an important property for random walks and also demonstrates the connection between Brownian motion and the normal distribution.\\

Let us re-introduce the duration of a time step $\tau$ take a limit as the step size $l$ and $\tau$ approach zero as $\ninfty$ is such a way that 
$$ {\sigma^2\over 2\tau} = D,  \quad nl(p-q) = x_0, \quad n\tau = t$$ where $D$ is a constant, $x_0$ is the mean position and $t$ is the total duration of $n$ steps. Taking the limit we find  
 \be \lim_{\ninfty} f_{X_n}(x) = {1\over\sqrt{4\pi Dt} } \exp\Big\{-{(x- x_0)^2 / 4Dt} \Big\} \label{eqn:walkEinstein} \ee 
Eq. [\ref{eqn:walkEinstein}] was derived by Albert Einstein in his original description of the probability law for the position of a physical particle undergoing Brownian motion  \cite{Einstein}.
The constant $D$ is known as the constant of diffusion and can be related to other physical constants in the case of Brownian Motion:
$$D = {2RT\over Nf},$$ where $R$ is the universal gas constant, $T$ is the absolute temperature, $N$ is the Avogadro number, and $f$ the friction coefficient specific to the particles undergoing Brownian motion. In light of the connection with the Random Walk, we see that Einstein's result emerges in probability theory as a consequence of the Central Limit Theorem. We will use Eq. [\ref{eqn:walkEinstein}] as one of the fundamental properties of the mathematical abstraction of Brownian Motion known as the Wiener Process. 


\section{Brownian Motion -- The Wiener Process}
\label{sec:Wiener}

The mathematical idealization of Brownian motion is called the Wiener process and it is an example of a stochastic process with stationary independent increments.  \\ 

A stochastic process $\{B(t), t\ge 0\}$ taking values in $R$ is said to be a Brownian motion in 1-dimension (a Wiener process) if\\

(a) $\{B(t), t\ge 0\}$ has stationary independent increments,\\
(b) For every $t>0, B(t)$ is normally distributed, \\
(c) For all $t>0, E[B(t)] = 0$\\
(d) $B(0) = 0$\\

Obviously the above stochastic process can be extended to $\mathbb{R}^n$ by allowing $B(t)$ to be vector-valued with $n$ components.\\

We use the notation $E[B(t)]$ to represent the expectation, or the mean, of the random number $B(t)$ averaged over the elementary outcomes $\omega$ in $\Omega$.\\

We will also use both notation $B_t(\omega)$ in place of $B(t)$ to emphasize the dependence of the Brownian motion on the particular elementary outcome $\omega \in \Omega$ that gives rise to it.\\

(Without loss of generality the above Brownian motion is assumed to start from the origin $0$. We could use any point $B(0) = x$ on the real number line as the starting point.)\\

Brownian motion can be considered as the integral with respect to time of ``white noise'' (more on this later). Because integration ``improves'' continuity properties representations of Brownian motion, $B(t)$ will have better continuity properties than ``white noise''. It appears to be reasonable to describe Brownian motion in terms of the observed values taken at discrete time intervals, say at the points $t_1, t_2, ..., t_n$ and hope to characterize it by finding a joint probability function for $B(t_1), B(t_2), ..., B(t_n)$.\\

Let us examine some of the consequences of discretizing the $t$ (or time) parameter space. Let us break up the interval $[0,t)$ into $[0,s) \cup [s,t)$.  This can be thought of as several points in order $0 \le s \le t$. \\

By (b) above, we see that $$E[B(t) - B(s)] = 0$$ because $E[B(t) - B(s)] = E[B(t)] - E[B(s)] = 0 - 0 = 0$\\

However more characteristics can be obtained by using properties (a) through (d) above. We use (d) out of convenience -- there is nothing sacred about setting the coordinate system up so that at $t=0$ we have $B(0) = 0$. It is a convenient choice that simplifies our derivations.\\

Now consider the time interval $0 \le s \le t$. We can assign values $t_1$ and $t_2$ to the lengths of the respective separations as 
\begin{eqnarray*}
s &=& t_1 \\
t &=& t_1 + t_2 \quad \hbox{and conversely} \\
t_1 &=& s \\
t_2 &=& t - s
\end{eqnarray*}

Notice that we can write $$B(t) = B(s) + B(t) - B(s)$$
Now $E[B(s)] = 0$ and $E[B(t) - B(s)] = 0$ for $t\ge s\ge0$ and using the property of stationary independent intervals we also have $$E[B(s) (B(t) - B(s))] = E[B(s)]\cdot E[B(t) - B(s)] = 0$$\\

Since $B(s)$ and $B(t) - B(s)$ for $t\ge s \ge 0$ are independent random variables we can express the variance $\hbox{Var}[B(t)]$ as $$\hbox{Var}[B(t)] = \hbox{Var}[B(s)] + \hbox{Var}[B(t) - B(s)]$$ 
Using the property of Stationary Independent Increments we see that $$\hbox{Var}[B(t) - B(s)] = \hbox{Var}[B(t-s) - B(0)] = \hbox{Var}[B(t-s)] = \hbox{Var}[B(t_2)]$$
Using this result and noting that $s = t_1$ and $t = t_1 + t_2$ we can write 
$$\hbox{Var}[B(t_1 + t_2)] = \hbox{Var}[B(t_1)] + \hbox{Var}[B(t_2)]$$ So that $f(t) = \hbox{Var}[B(t)]$ satisfies the property that 
$$f(t_1 + t_2) = f(t_1) + f(t_2)$$ One can easily extend this by induction to $$f(\sum_{i=1}^n\, t_i) = \sum_{i=1}^n f(t_i)$$ Consider a rational number $t = n/m$ for integers $n$ and $m$. Then $$f(t) = f(n/m) = \sum_{i=1}^n\, f(1/m) = n\cdot f(1/m)$$ Also, $f(1) = f(m/m) = m\cdot f(1/m)$ and hence $$f(1/m) = (1/m)\cdot f(1).$$ Putting these facts together, we obtain $$f(t) = f(n/m) = (n/m) f(1).$$ Now if we represent $f(1)$ by the constant $c$, then $$f(t) = f(n/m) = (n/m)\cdot f(1) = c\cdot t.$$ Since $\hbox{Var}[X] \ge 0$ for any random variable $X$, then $c>0$ which we will denote by $c=\sigma^2$.\\

Using these facts we find that $\hbox{Var}[B(t)] = \sigma^2 t$ and therefore
$$\hbox{Var}[B(t)] = \hbox{Var}[B(s)] + \hbox{Var}[B(t) - B(s)]$$ can be written as $$\hbox{Var}[B(t) - B(s)] = \sigma^2(t-s)$$ when $t\ge s\ge0$. The general case can be written as 
\be \hbox{Var}[B(t) - B(s)] = \sigma^2|t-s|. \label{eqn:variance}\ee

Setting $s=0$ we obtain $$\hbox{Var}[B(t)] = E[(B(t))^2] - (E[B(t))^2 = E[B(t)^2] = \sigma^2t.$$

Another consequence of Stationary Independent Increments using $B(t) = B(s) + B(t) - B(s)$ is that

\begin{eqnarray*}
E[B(s)B(t)] &=& E[B(s) \cdot (B(s) + B(t) - B(s)) ]\\
                  &=& E[B^2(s)] + E[B(s)\cdot  (B(t) - B(s)]\\
                  &=& E[B^2(s)] = \sigma^2 s
\end{eqnarray*}
since $E[B(s)\cdot (B(t) - B(s) ) ] = 0$ for $t\ge s\ge 0$. Since we assumed that $t\ge s\ge 0$ in the derivation we can summarize the general case as 
\be E[(B(t) B(s)] = \sigma^2 \hbox{min}(s,t) \label{eqn:correlation}\ee\\

The parameter $\sigma^2$ above characterizes the normal distribution referred to in part (b) above. If we take only one coordinate $x$ of the Brownian path and consider the motion as a function of time, then taking $\sigma=1$ normalizes the motion to fit the time scale. In this case we can develop a probability distribution for $x(t)$. Let us take $x(t) = 0$ at $t=0$ and consider the position of the particle seen at set of times $0 \le t_1 \le  t_2 \le \hdots \le t_n$. The assuming the time scale is chosen so that $\sigma=1$, then the probability that the particle lies between $x_1$ and $x_1 + dx_1$ at time $t_1, \hdots,$ between $x_n$ and $x_n + dx$ at time $t_n$ is \cite{Wiener1}
$${ \exp\left[ -{x_1^2\over2t_1} - {(x_2-x_1)^2\over2(t_2-t_1)} - \hdots - {(x_n - x_{n-1})^2\over2(t_n - t_{n-1})}  \right] \over \sqrt{|(2\pi)^n t_1(t_2-t_1)\hdots (t_n-t_{n-1})|  } }\, dx_1\hdots dx_n$$ \\ The above probability measure was developed by Norbert Wiener in 1921 and represents an advance over the early work of Albert Einstein on Brownian motion \cite{Einstein}.\\

In what follows we will examine further properties of Brownian Motion and Ito's Integral. It is occasionally useful to define two stochastic processes which are versions of each other: 

\begin{definition}
Suppose that $\{ X(t,\omega) \}$ and $\{ Y(t,\omega) \}$ are stochastic processes on $(\Omega, {\cal F}, P)$. Then we say that $\{ X(t) \}$ is a version of $\{ Y(t) \}$ if 
$$P(\{\omega: X(t,\omega) = Y(t,\omega) \}) = 1\quad\hbox{for all t.}$$ Thus if $X(\omega,t) \in \mathbb{R}^n$, then from the point of view that a stochastic process is a probability law on $(\mathbb{R}^n)^{[0,\infty)}$ two such 
processes are indistinguishable, but nevertheless their path properties may be different. Two stochastic processes that are different versions of each other can be thought of as belonging to the same equivalence class (see Section [\ref{sec:equivalence}]).
\end{definition}

The properties of Brownian motion will be discussed further in the next section, but in passing, please note the following famous theorem of Kolmogorov (see \cite{Baudoin}, \cite{vanZanten} and \cite{Oskendal1}).
\begin{theorem}{(Kolmogorov's continuity theorem)} Suppose that the process $X = \{ X(t), t\ge 0\}$ satisfies the following condition: For all $T>0$ there exists positive constants $\alpha, \beta, D$ such that 
\be E[|X(t) - X(s)|^\alpha] \le D\cdot |t-s|^{1+\beta}; 0\le s, t \le T. \label{eqn:continuity}\ee 
Then there exists a continuous version of $X(t)$.
\end{theorem} 

\begin{proof}
We follow van Zanten's proof\footnote{See Harry van Zanten, \cite{vanZanten}, section [1.3]}.
First observe that by Chebyshev's inequality, Eq. [\ref{eqn:continuity}] implies that the process $X(t)$ is continuous in probability. 
Without loss of generality, we assume that $t \in [0,T]$ with $T=1$ and work with subintervals, $\Delta_n$, made up of the dyadic rationals: $\Delta_n = {k/2^n: k=0,1,\hdots ,2^n}$ and let 
$\displaystyle{D = \bigcup\limits_{n = 1}^{\infty} \Delta_n}$. Then $D$ is a countable set, and $D$ is dense\footnote{$\overline{A}$ is the closure of $A$, i.e., the set $A$ together with all its limit point (see \cite{Protter}, page 366): Let A be a  metric space. We say a set $A \subset S$ is {\it dense}\/ in S iff $\overline{A} = S $} in $[0, 1]$. Our next aim is to show that with probability 1, the process $X(t)$ is uniformly continuous on $D$. Pick any $\gamma \in [0, \beta/\alpha]$. Using Chebyshev's inequality again, we see 
$$P( |X(k/2^n) - X((k-1)/2^n) | \ge 2^{-\gamma n} ) \le {K\over 2^{n(1+\beta - \alpha\gamma)}}$$for some positive constant $K$.
It follows that
$$ P(\max_{1\le k \le2^n} |X(k/2^n) - X((k-1)/2^n) | \ge 2^{-\gamma n} ) \le \sum_{k=1}^{2^n} P( | X(k/2^n) - X((k-1)/2^n) | \ge 2^{-\gamma n}) \le {K\over 2^{n(\beta -\alpha\gamma)} }$$
 
Hence, by the Borel-Cantelli lemma (see Section [\ref{sec:BorelCantelli}]), there almost surely exists an $N \in \mathbb{N}$ such that
\be \max_{1\le k \le2^n} |X(k/2^n) - X((k-1)/2^n) | \le 2^{-\gamma n}  \label{eqn:dyadicMax}\ee for all $n \ge N$.
Next, consider an arbitrary pair $s,t \in D$ such that $0 < t-s < 2^{-N}$. 
We aim to show that 
\be | X(t) - X(s) | \le K |t-s|^\gamma, \label{eqn:induction}\ee for some positive constant $K$.
 There exists an $n \ge N$ such that $2^{-(n+1)} \le  t - s < 2^{-n}$. We claim that if 
$ s,  t \in \Delta_m$ for $m \ge n+1$, then
\be  | X(t) - X(s) |  \le 2\sum_{k=n+1}^m 2^{-\gamma k} \label{eqn:twosum}\ee

To see this we use induction. Suppose first that $s, t \in  \Delta_{n+1}$. Then necessarily, $t=k/2^{n+1}$ and $s = (k-1)/2^{n+1}$ for some 
$k \in (1,\hdots,2^{n+1}$. By Eq. [\ref{eqn:dyadicMax}], it follows that
$$  | X(t) - X(s) |  \le 2^{-\gamma (n+1)} $$
 which proves the claim for $m = n + 1$. Now suppose that it is true for $m = n+1, \hdots, l$ 
and assume that $s, t \in \Delta_{l+1}$. Define the numbers $s', t' \in \Delta_l$ by $s' = \min\{u \in \Delta_l : u\ge s\}, t' = \max\{u\in \Delta_l : u \le t\}.$
Then by construction, $s\le s' \le  t' \le t$ and $s' - s \le 2^{-(l+1)}, t' - t \le 2^{-(l+1)}.$ Hence, by the triangle inequality, Eq. [\ref{eqn:dyadicMax}] and the induction hypothesis,
\begin{eqnarray*}
| X(t) - X(s) | &\le& |X(s') - X(s) | + |X(t') - X(t) | + |X(t') - X(s')| \\
&\le& 2^{-\gamma(l+1)} + 2^{-\gamma(l+1)} + 2\sum_{k=n+1}^l 2^{-\gamma k} = 2\sum_{k=n+1}^{l+1} 2^{-\gamma k},
\end{eqnarray*}
 so the claim is true for $m = l + 1$ as well. The proof of Eq. [\ref{eqn:induction}] is now straightforward. 
 Indeed, since $t, s \in \Delta_m$ for some large enough $m$, relation Eq. [\ref{eqn:twosum}] implies that
 $$|X(t) - X(s) | \le 2\sum_{k=n+1}^\infty = {2\over 1-2^{-\gamma}} 2^{-\gamma(n+1)} \le {2\over 1- 2^{-\gamma}}|t-s|^\gamma$$
%  ?Xs?�2 ??� 2??k = 2 2??(n+1) � 2 |t?s|?. k=n+1 1?2?? 1?2??
Observe that Eq. [\ref{eqn:induction}] implies in particular that, almost surely, the process $X(t)$ is uniformly continuous on $D$. 
  In other words, we have an event $O \in \Omega$ with $P(O) = 1$ such that for all $\omega \in  O$, the sample path $X(t, \omega)$ is uniformly
continuous on the countable, dense set $D$. Now we define a new stochastic process 
$$Y(t) = \Bigg\{
\begin{array}{ll}
X(t) &\hbox{if $t\in D$} \\
\lim_{t_n \to t, t_n \in D} X(t_n, \omega) &\hbox{if $t \not\in D$.} 
\end{array}
$$
 The uniform continuity of $X(t)$ implies that $Y(t)$ is a well-defined, continuous stochastic process. 
Since $X(t)$ is continuous in probability (see the first part of the proof), $Y(t)$ is a version of $X(t)$.\qed\end{proof}

As an example, we will apply Kolmogorov's continuity theorem to $n$-dimensional Brownian motion: $B(t) = [B^{(1)}(t), B^{(2)}(t), \hdots, B^{(n)}(t)]$, where the components $B^{(j)}(t), 1 \le j \le n$ are independent $1$-dimensional Brownian motions. \\

\begin{example}{(\elevenit Brownian motion exists)}\/

Let $B(t)$ be a standard $1$-dimensional Brownian motion on $\mathbb{R}$ with $B(0) = 0$ and $\sigma^2 = 1$. 
The moment generating function (see Section [\ref{sec:generating}]) is given by 
\be E[ e^{u B(t)} ] = \exp({1\over2} u^2 t) \quad \hbox{for all $u\in \mathbb{R}^n.$} \ee

From this we can conclude that 

\be E[ e^{u B(t)} ] = \sum_{k=0}^\infty {u^k E[B^k(t)]\over k!}  = \sum_{j=0}^\infty {t^j u^{2j} \over 2^j j!} \ee
Hence 
$$E[B^{2k+1}(t)] = 0 \quad\hbox{and}\quad E[B^{2k}(t)] = {(2k)! t^k\over k! 2^k}$$ for $k=0,1, \hdots$. Note that all the odd powers of $k$ give zero expectations. \\

We now extend the above argument to Brownian motion in $\mathbb{R}^n:$  

\begin{eqnarray*}    
E[| B(t) - B(s) |^4]  &=& \sum_{i=1}^n E[(B^{i} (t) - B^{i} (s))^4 + \sum_{i\ne j} E[ (B^{(i)} (t) - B^{(i)}(s))^2 (B^{(j)} (t) - B^{(j)}(s))^2 ]  \\
&=& n\cdot {4!\over4\cdot2!}\cdot (t-s)^2 + n(n-1)(t-s)^2 \\
&=& n(n+2)(t-s)^2
\end{eqnarray*}
Hence we see that $n$-dimensional standard Brownian motion satisfies the criterion for Kolmogorov's continuity theorem with $\alpha = 4, \beta = 1$ and $D=n(n+2)$ and therefore it 
has a continuous version. We will look at this property again from a different point of view based on mean-square calculus and an intuitive approach representing Brownian motion as the integral of ``white noise''. Kolmogorov's continuity theorem is a powerful method.
\end{example} 



\section{Mean-Square Calculus}
\label{sec:MeanSquareCalculus}
 
 Please see \cite{Stensby} and \cite{Jazwinski} for a general description of mean-square Calculus. \\
 
 As mentioned in Section [\ref{sec:Probability}], mean-square (m.s.) calculus is used simplify many of the derivations but still keep the subject general enough to describe a large body of the interesting subjects in stochastic processes. The main cornerstones on which mean-square calculus stands are the ideas of {\elevenit mean-square limit, mean-square continuity, mean-square differentiability, and mean-square integrability.} The concept of mean-square convergence appears in many other areas such as the inner product spaces (Hilbert space) of Quantum Mechanics.\\ 
 
Earlier we defined the correlation function for a stochastic process by $\Gamma(t_1, t_2) = E[X(t_1) X(t_2)] $ The correlation function can also be thought of as a kind of inner product 
$$\langle X(t_1), Y(t_2) \rangle = E[X(t_1) X(t_2)]$$

The inner product exists as a real number for every $X(t_1)$ and $Y(t_2)$ which have finite average power and satisfies the following rules:\\

1.) $\langle X(t_1), X(t_1) \rangle \ge 0$ and $\langle X(t_1), X(t_2) \rangle = 0$ iff $X(t) = 0$ almost surely (i.e., $P[X(t) = 0] = 1$),\\ 
2.) $\langle X(t_1), Y(t_2) \rangle = \langle Y(t_2), X(t_1) \rangle $,\\ 
3.) $\langle c X(t_1), Y(t_2) \rangle = c \langle X(t_1), Y(t_2) \rangle$, where $c \in \mathbb{R}$.\\

The equivalence of $\langle X(t), X(t) \rangle = 0$ and $X(t) = 0$ identically is a general requirement of an inner product, but this is not true in the above case since we only have $X(t) = 0$ almost surely (wp1). There are ways around this subtle point. For example we can use $\langle , \rangle = 0$ to define an equivalence relation (see Section [\ref{sec:equivalence}]) between $X(t)$ and $Y(t)$ and say that $X R Y$ if $\langle X(t) - Y(t) , X(t) - Y(t) \rangle = 0$ and then take the continuous version from this equivalence class as the representative version (assuming it exists). With this understanding we define the {\elevenit vector norm} of a vector-valued process similarly:
$$\| X(t) \| = \sqrt{\langle X(t), X(t) \rangle}$$

This vector norm satisfies:\\

1.) $\| X(t) |\ \ge 0$ and $\| X(t) \|= 0$ iff $X(t) = 0$ almost surely (i.e., $P[X(t) = 0] = 1$),\\ 
2.) $\| c X(t) \| = c \| X(t) \| $, where $c \in \mathbb{R}$.\\
3.) $\|  X(t) +Y(t) \|  \le \| X(t) \| + \| Y(t) \|$ \quad \hbox{(the triangle inequality)}. \\ 
 
 If $E[X(t)] = 0$, then $\| X(t) \|$ is the standard deviation of $X(t)$.\\

We assume that the stochastic processes of interest have finite average power $$E[X^2(t)] < \infty$$ for all t. We also assume, without loss of generality, that all finite-power processes have zero mean. Since the assumption of $E[X^2(t)] < \infty$ implies that $E[X] < \infty$ by the Cauchy-Schwarz inequality we can transform variables to 
$$Y(t) = X(t) - E[X]$$ that have zero mean. \\

\subsection{Mean-Square Limit}

The various kinds of convergences used are summarized in Section [\ref{sec:convergence}]. Fundamental to the subject of mean-square limits is the Cauchy convergence theorem: 

\begin{theorem}
Let $X(t)$ be a real-valued, finite power stochastic process. The mean-square limit
\[ Y(t) = \mslim_{t'\to t} X(t') = \mslim_{\epsilon \to 0} X(t+\epsilon) \]
exists as a unique (in m.s. sense) stochastic process iff 
\[ \lim_{t_1, t_2 \to t} E\Big[ [X(t_1) - X(t_2) ]^2  \Big] = \lim_{\epsilon_1, \epsilon_2 \to 0} E\Big[  [X(t+\epsilon_1) - X(t+\epsilon_2)]^2  \Big]  = 0 \]
\end{theorem}

\begin{definition}
The mean-square limit $\displaystyle{\mslim_{\ninfty} X(t+t_n)}$ exists as a unique stochastic process iff the double limit
\[ \mslim_{n, m \to \infty} [X(t+t_n) - X(t + t_m)]  = 0 \]
\end{definition}

Because we are working with zero-mean processes with finite average power the correlation function is finite by the Cauchy-Schwarz inequality:
\[ | \Gamma(t_1, t_2) | = | E[X(t_1) X(t_2)] | \le \sqrt{E[X^2(t)] E[X^2(t_2)]} < \infty \]

%Continuity properties for stochastic process depend on the type of convergence methods employed.

\subsection{Mean Square Continuity}

\begin{definition}
A stochastic process $X(t)$ is {\elevenit mean-square (m.s.)  continuous}\/ at time $t$ if 
\[ X(t) = \mslim_{t'\to t}[X(t') - X(t)] \equiv \mslim_{\epsilon \to 0} [ X(t+\epsilon) - X(t) ] = 0 \]
\end{definition}

Another way to express this limit is 
\[ \lim_{t' \to t} \| X(t') - X(t) \|^2  = \lim_{t'\to t} E\left[(X(t') - X(t))^2 \right] = \lim_{\epsilon\to 0}E\left[(X(t+\epsilon) - X(t))^2 \right]  = 0   \]

Mean-square continuity does not imply continuity at the individual sample function level. 

The condition for mean-square continuity is given by the following theorem:
\begin{theorem}
At time t, a stochastic process $X(t)$ is mean-square continuous iff the correlation function $\Gamma(t_1, t_2)$ is continuous at $t=t_1=t_2$.
\end{theorem}

\begin{proof}
Continuity of $\Gamma(t_1, t_2)$ is sufficient for m.s. continuity of $X(t)$:
\begin{eqnarray*} 
E[ (X(t') - X(t))^2 ] &=& E[X^2(t')] - E[X(t')X(t)] - E[X(t)X(t')] + E[X^2(t)] \\
&=& \Gamma(t', t') - \Gamma(t', t) - \Gamma(t, t') + \Gamma(t, t)  
\end{eqnarray*}
Therefore if $\Gamma(t_1, t_2)$ is continuous at $t = t_1 = t_2$, then the RHS of the above equation is equal to zero as $t' \to t$ and $X(t)$ is mean-square continuous at $t$.\\

Now let us assume that $X(t)$ is mean-square continuous at $t$. Form the difference $\Gamma(t_1, t_2) - \Gamma(t, t)$:

\begin{eqnarray*} 
\Gamma(t_1, t_2) - \Gamma(t,t) &=&  E[X(t_1) X(t_2)] - E[X(t) X(t) ]\\
&=& E[ (X(t_1) - X(t)) (X(t_2) - X(t))] + E[(X(t_1) - X(t))X(t)] \\
&=& E[X(t) ( X(t_2) - X(t))]
\end{eqnarray*} and therefore
\begin{eqnarray*} 
|\Gamma(t_1, t_2) - \Gamma(t,t)| &\le& |E[ (X(t_1) - X(t)) (X(t_2) - X(t))]|  + |E[(X(t_1) - X(t))X(t)]| \\
&+& |E[X(t) ( X(t_2) - X(t))]|
\end{eqnarray*}

Apply the Cauchy-Schwarz inequality to each term on the RHS of the above to obtain
\begin{eqnarray*} 
|\Gamma(t_1, t_2) - \Gamma(t,t)| &\le& \sqrt{ E[ (X(t_1) - X(t))^2 E(X(t_2) - X(t))^2] }  + \sqrt{E[(X(t_1) - X(t))^2 E[X^2(t)]} \\
&+& \sqrt{ E[X^2(t)] E( X(t_2) - X(t))^2]}
\end{eqnarray*} Since $X(t)$ is m.s. continuous at $t$, the RHS of the above approaches zero as $t_1, t_2 \to t$ as follows
\[ \lim_{t1, t2 \to t} \Gamma(t_1, t_2) = \Gamma(t,t) \]and $\Gamma(t_1, t_2)$ is continuous at $t = t_1 = t_2$.\qed\end{proof}

\subsection{Mean-Square Differentiation}

We introduce the following operator on functions of one variable to simplify the expressions which follow:

$$\Delta_\epsilon f(x)  = f(x+\epsilon) - f(x).$$ Because the operations which follow involve functions of two variables we define two delta-operators
\begin{eqnarray*}
\Delta_\epsilon^{(1)} f(t,s) &=& f(t+\epsilon, s) - f(t,s) \\ 
\Delta_\epsilon^{(2)} f(t,s) &=& f(t, s +\epsilon) - f(t,s) \\ 
\end{eqnarray*}

A stochastic process $X(t)$ has a mean-square (m.s.) derivative denoted by $\dot{X}(t)$ if there exists a finite power stochastic process

\[\dot{X}(t) = \mslim_{\epsilon\to0} \left[ {X(t+\epsilon) - X(t) \over \epsilon}  \right] =\mslim_{\epsilon\to0}\left[ {\Delta_\epsilon X(t)\over \epsilon} \right]    \] 

These equations can also be expressed as

\[ \lim_{\epsilon\to0} E \left[ \left( {X(t+\epsilon) - X(t) \over \epsilon} - \dot{X}(t) \right)^2 \right] =\lim_{\epsilon\to0}E\left[ \left( {\Delta_\epsilon X(t)\over \epsilon} - \dot{X}(t) \right)^2 \right] = 0 \] 

According to the Cauchy Convergence Theorem, as $\epsilon\to0$, the quantity $\Delta_\epsilon X(t) / \epsilon$ has a m.s. limit iff
\[ \mslim_{\epsilon_1, \epsilon_2\to 0} \left({\Delta_{\epsilon_1} X(t)\over \epsilon_1}   - {\Delta_{\epsilon_2} X(t)\over \epsilon_2}   \right) = 0.  \]
 
This condition can be written as 

\begin{eqnarray*} 
\lim_{\epsilon_1, \epsilon_2\to 0}E\left[ \left({\Delta_{\epsilon_1} X(t)\over \epsilon_1}   - {\Delta_{\epsilon_2} X(t)\over \epsilon_2}   \right)^2\right] &=&
\lim_{\epsilon_1, \epsilon_2\to 0}E\left[ \left({\Delta_{\epsilon_1} X(t)\over \epsilon_1}\right)^2   - 2\left({\Delta_{\epsilon_1} X(t)\over \epsilon_1}\right) \left( {\Delta_{\epsilon_2} X(t)\over \epsilon_2}   \right)
+ \left({\Delta_{\epsilon_2} X(t)\over \epsilon_2}\right)^2\right] \\ 
&=& 0.
\end{eqnarray*}

There are two terms in the above which are equal to 
\begin{eqnarray*} 
\lim_{\epsilon \to 0}E\left[ \left({\Delta_{\epsilon} X(t)\over \epsilon}\right)^2\right] &=& \lim_{\epsilon \to 0}E\left[ \left({ X(t+\epsilon) - X(t) \over \epsilon}\right)^2\right]\\
&=& \lim_{\epsilon \to 0} {\Gamma(t+\epsilon, t+ \epsilon) - \Gamma(t+\epsilon, t) - \Gamma(t, t+\epsilon) + \Gamma(t,t) \over \epsilon^2}
\end{eqnarray*}
and a cross term that is equal to 
\begin{eqnarray*} 
\lim_{\epsilon_1, \epsilon_2 \to 0}E\left[ \left({\Delta_{\epsilon_1} X(t)\over \epsilon_1}\right) \left({\Delta_{\epsilon_2} X(t)\over \epsilon_2}\right)  \right] &=& 
\lim_{\epsilon_1, \epsilon_2 \to 0}E\left[ \left({ X(t+\epsilon_1) - X(t) \over \epsilon_1}\right) \left({ X(t+\epsilon_2) - X(t) \over \epsilon_2}\right)   \right]\\
&=& \lim_{\epsilon_1, \epsilon_2 \to 0} {\Gamma(t+\epsilon_1, t+ \epsilon_2) - \Gamma(t+\epsilon_1, t) - \Gamma(t, t+\epsilon_2) + \Gamma(t,t) \over \epsilon_1\epsilon_2}
\end{eqnarray*}

Substituting these results back into the expression for the Cauchy series above we obtain
\begin{eqnarray*} 
\hbox{\bf (A)}\quad \lim_{\epsilon_1, \epsilon_2\to 0}E\left[ \left({\Delta_{\epsilon_1} X(t)\over \epsilon_1}   - {\Delta_{\epsilon_2} X(t)\over \epsilon_2}   \right)^2\right] &=&
2 \lim_{\epsilon \to 0} {\Gamma(t+\epsilon, t+ \epsilon) - \Gamma(t+\epsilon, t) - \Gamma(t, t+\epsilon) + \Gamma(t,t) \over \epsilon^2} \\
&\null& -2\lim_{\epsilon_1, \epsilon_2 \to 0} {\Gamma(t+\epsilon_1, t+ \epsilon_2) - \Gamma(t+\epsilon_1, t) - \Gamma(t, t+\epsilon_2) + \Gamma(t,t) \over \epsilon_1\epsilon_2}.
\end{eqnarray*}

Putting all these conditions together leads us to the 
\begin{theorem}\label{thm:msDifferentiable}
A finite-power stochastic process $X(t)$ is mean-square differentiable at $t$ iff, the double limit
\[ \lim_{\epsilon_1, \epsilon_2\to 0} { \Delta^{(1)}_{\epsilon_1} \Delta^{(2)}_{\epsilon_2} \Gamma(t,t) \over \epsilon_1 \epsilon_2} =
\lim_{\epsilon_1, \epsilon_2 \to 0} {\Gamma(t+\epsilon_1, t+ \epsilon_2) - \Gamma(t+\epsilon_1, t) - \Gamma(t, t+\epsilon_2) + \Gamma(t,t) \over \epsilon_1\epsilon_2} \]
exists and is finite. 
\end{theorem}

\begin{proof}
Sufficient Condition: if ${\partial \Gamma(t_1, t_2)\over \partial t_1},  {\partial \Gamma(t_1, t_2)\over \partial t_2},$ and ${ \partial^2\Gamma(t_1, t_2)\over \partial t_1 \partial t_2}$ exist in a neighborhood of $(t_1, t_2) = (t,t)$ and ${\partial^2  \Gamma(t_1, t_2) \over \partial t_1 \partial t_2}$ is continuous at $(t_1, t_2) = (t, t)$, then the above limit exist by the Calculus, and the process $X(t)$ will be differentiable at $t$.\\

If the above sufficient condition does not hold, then $X(t)$ is m.s. differentiability implies that the above mean-square expectation {\bf (A)} is zero, independently of how $\epsilon_1$ and $\epsilon_2$ approach zero which means that the right-hand-side of the expression in the above exists and is finite. On the other hand if the right-hand-side of the expression in theorem [\ref{thm:msDifferentiable}] exists and has the value $R$, then this must hold regardless of how $\epsilon_1$ and $\epsilon_2$ approach zero. That means the limit of the first term on the RHS of the expectation {\bf (A)} is 2R and therefore the two terms cancel out exactly and $X(t)$ is m.s. differentiable.\qed
\end{proof}

Assuming that the sufficient condition in the 1st part of the above proof holds, the correlation function for the m.s. derivative $\dot{X}(t)$ is
$$\Gamma_{\dot{X}} (t_1, t_2) = E[\dot{X}(t_1) \dot{X}(t_2)]  = {\partial^2 \Gamma(t_1, t_2)\over \partial t_1 \partial t_2}.$$

\begin{example}Brownian Motion. We saw that the correlation function for Brownian motion is given by $\Gamma(t,s) = E[ B(t) B(s)] = \sigma^2 \min(s,t)$. \\

Now 
$$
\min(s,t) = \Big\{
\begin{array}{cc}
t, & t < s \\
s, & t > s
\end{array}
$$so that 
$${\partial\over \partial s} \min(s,t) = 
\begin{array}{cc}
0, & t < s \\
1, & t > s
\end{array}$$which is the Heavyside unit step function (in $t$) and its derivative with respect to $t$ is the Dirac delta function: $\delta(t - s)$. \\

The correlation function for the m.s. derivative of Brownian motion is therefore 
$${\partial^2 \Gamma(t,s) \over \partial t \partial s}  = \sigma^2 \delta( t-s ).$$ Hence the second derivative does not exist in the sense of ordinary calculus function and therefore Brownian motion is not m.s. differentiable (we will see more on this in Section [\ref{sec:Brownian}]). However, in the sense of {\elevenit generalized functions}\/ we see that the second derivative of the correlation function is the delta-function which is the correlation function for the ``white noise'' stochastic process. Hence in the sense of generalized functions, Brownian motion is the integral of white noise, even though it is not m.s. differentiable. 
\demo\end{example}

\subsection{Mean-Square Riemann Integration}

The last subject we will explore in mean-square calculus is mean-square integration. Integrals of stochastic process appear in many applications -- for example a slowly varying signal might be corrupted by high-frequency noise. In another application a more realistic model for physical Brownian motion (which is a finite length path) is given by the Langevin equation\cite{Sjogren},\cite{CalTech} which requires integrating ``white noise'' in the presence of frictional effects such as viscosity. Integrals of stochastic processes will therefore enter in the construction of {\elevenit Filters}\/ for {\elevenit Smoothing} as well as {\elevenit Prediction}.\\% Filters are a future topic of these notes.\\

As in ordinary calculus we will partition the finite interval $[a, b]$ into subdivisions labeled by points $t_k, k = 0, 1, \hdots, n$ such that $$a = t_0 < t_1 < \hdots < t_n = b.$$ We will also label the increments in $t$ as $$\Delta t_i = t_i - t_{i-1}, \quad 1\le i \le n$$ We will label such a partition by $P_n$ which is made up of the $n+1$ points $t_k, k = 0, 1, \hdots n$. Also, although for simplicity we can take all the increments to be of equal size, we allow for the general case and define the upper bound on the increment (or mesh) size by
$$\Delta_n = \max_k \Delta t_k$$ 

As the partitions are refined into more and more points, $\Delta_n$ will decrease in a well behaved manner. 

Now, as in ordinary calculus we can form a Riemann sum for a finite-power stochastic process $X(t)$ as 

$$\sum_{k=1}^n X(t^*_k) \Delta t_k, \quad\hbox{where~~} t^*_k \in (t_{k-1}, t_k].$$

From this starting point we define the mean-square Riemann integral on the interval $[a, b]$ (known as the Wiener integral) as

\be \int_a^b X(t) dt \equiv \mslim_{\Delta_n \to 0} \sum_{k=1}^n X(t^*_k)\Delta t_k.\label{eqn:RiemannIntegral} \ee

If everything is well behaved, then as $\Delta_n\to0$, then $\ninfty$ and the Riemann sum converges in mean-square to the mean-square Riemann integral. We state the requirement for convergence to occur:

\begin{theorem}
The Wiener integral, Eq. [\ref{eqn:RiemannIntegral}], exists, iff the ordinary double integral 
\be \int_a^b \int_a^b \Gamma(s,t) ds dt \ee exists and is finite. 
\end{theorem}  

\begin{proof}We will apply the Cauchy Convergence Criteria to prove the above result. 
Let $P_n$ and $P_m$ denote two distinct partitions of the $[a, b]$ interval. Let us take them to be given by

$$P_n:  \Bigg\{
\begin{array}{rl}
a &= t_0 < t_1 < \hdots < t_n = b\\
\Delta t_i &= t_i - t_{i-1} \\
\Delta_n &= \max_{i} \Delta t_i
\end{array} $$

$$P_m:  \Bigg\{
\begin{array}{rl}
a &= s_0 < s_1 < \hdots < s_n = b\\
\Delta s_i &= s_i - s_{i-1} \\
\Delta_m &= \max_{i} \Delta s_i
\end{array} $$
According to the Cauchy Convergence Criteria, the Wiener integral (Eq. [\ref{eqn:RiemannIntegral}] ) exists iff, 

\be  \lim_{\Delta_n, \Delta_m \to 0} E \left[\left( \sum_{k=1}^n X(t^*_k) \Delta t_k - \sum_{j=1}^m X(s^*_j) \Delta s_j \right)^2 \right] = 0 \label{eqn:WienerConvergence}\ee

Now expand the square in Eq. [\ref{eqn:WienerConvergence}] and take the expectations to obtain 

\be 
\sum_{k=1}^n \sum_{i=1}^n \Gamma(t^*_k, t^*_i) \Delta_{t_k} \Delta_{t_i}  
- 2\sum_{k=1}^n \sum_{j=1}^m \Gamma(t^*_k, s^*_i) \Delta_{t_k} \Delta_{s_j} +   
\sum_{j=1}^m \sum_{i=1}^m \Gamma(s^*_k, s^*_i) \Delta_{s_k} \Delta_{s_i}  
= 0 \label{eqn:crossTerm}\ee
 
 As $\Delta_n$ and $\Delta_m$ approach zero, Eq. [\ref{eqn:WienerConvergence}] is true iff
 
 \be \lim_{\Delta_n, \Delta_m \to 0} \sum_{k=1}^n \sum_{j=1}^m \Gamma(t^*_k, s^*_i) \Delta_{t_k} \Delta_{s_j}  =  \int_a^b \int_a^b \Gamma(t,s) dt ds  \ee

The cross-term in Eq. [\ref{eqn:crossTerm}] must converge independent of the paths that $n$ and $m$ take as $n, m \to \infty$. If this happens the first and third sums on LHS of Eq. [\ref{eqn:crossTerm}] converge to the same double integral, and Eq. [\ref{eqn:WienerConvergence}] holds true.\qed 
 \end{proof}
 
 We end this section with an example based on Brownian motion. 
 \begin{example}
 Let $X(t)$ be the Brownian motion process, and consider the m.s. integral
 \be Y(t) = \int_0^t X(s) ds \ee
Now, for the Brownian motion process $\Gamma(u,v) = \sigma^2\min(u,v)$ and this can be integrated to obtain
\begin{eqnarray*}
\int_0^t \int_0^t \Gamma(u,v) du dv &=& \int_0^t \int_0^t \sigma^2\min(u,v) du dv \\
&=& \sigma^2 \int_0^t [\int_0^v  u du + \int_v^t v du]\, dv \\
&=& \sigma^2 t^3/3
\end{eqnarray*}
Therefore, Brownian motion is m.s. Riemann integrable. 
 \demo\end{example}
 
\section{Properties of Brownian Motion Paths}
\label{sec:Brownian}

Most of the ideas in this section were explained to me by Howard Weiner of UC Davis \cite{Weiner1}.\\

There are several important properties of Brownian motion:\\

1) Brownian motions have finite total Quadratic Variation.\\
2) Brownian motions have infinite length almost surely (a.s.) \\
3) Brownian motions are non-differentiable \\
4) Brownian motions are continuous functions $t$ or have continuous versions.\\

In what follows in this section let B(t) be the standard Brownian Motion (B.M.) with mean zero and variance parameter $\sigma^2 = 1$.\\

\subsection{Brownian motions have finite Quadratic Variation}

The exposition in this section works with one complete Brownian motion chosen from the sample space at a time, or at the sample function level. \\

Define the Quadratic Variation of B.M. as follows (on $[0,1]$):

Let $$V_n \equiv \sum_{l=1}^{2^n}\, \left[ B({l\over2^n}) - B({l-1\over2^n})\right]^2,$$ where we again sum over the $2^n$ dyadic intervals of length $1/2^n$.\\

By stationary independent increments, the random variables $$\Big[ B({l\over2^n}) - B({l-1\over2^n}) \Big]$$ are independent identically distributed (IID) $\sim N(0,1/2^n)$. Therefore $E[V_n] = 1$.\\

From Eq.[\ref{eqn:moments}], $X\sim N(0,\sigma^2) \implies E[X^4] = 3\sigma^4$ so that $$\hbox{Var}[X^2] = E[X^4] - (E[X^2])^2 - 3\sigma^4 - \sigma^4 = 2\sigma^4.$$ Then 

$$\hbox{Var}[V_n] = E[(V_n - 1)^2] = \left(2({1\over2^n})^2\right) \cdot 2^n = {1\over2^{n-1}}$$ 

Therefore $$\lim_{\ninfty} E[(V_n - 1)^2]  \goes 0\quad\hbox{and}\quad\lim_{\ninfty}\, V_n = 1\quad\hbox{in q.m.}$$
Then define the Quadratic Variation of BM on $[0,1]$ by $$V \equiv \lim_{\ninfty}\, V_n.$$ Therefore $V\goes1$ on $[0,1]$ in q.m.\\

To go a little further\footnote{Following the discussion in \url{https://idontgetoutmuch.wordpress.com/2012/03/17/the-quadratic-variation-of-brownian-motion}} define a corresponding set of random variables $X_l$ by
$$X_l = \left[ B(t_{l+1}) - B(t_l)\right]^2 - [t_{l+1} - t_l]$$ and use the partition, $\pi_n$ of $[0,1]$ given by $t_l = {l\over2^n}, l = 0,1,2,\hdots 2^n,  \Delta t_n =  2^{-n}$.

Then $$E\Big[ \left( \sum_{i=0}^{n-1} \left[ B(t_{l+1}) - B(t_l)\right]^2 - (t_{l+1} - t_l) \right)^2\Big] = E\Big[ \left( \sum_{i=0}^{n-1} \left[ B(t_{l+1}) - B(t_l)\right]^2 - t]\right)^2\Big]$$ 
Since the $X_l$'s are independent with zero means we can write the above as
$$E\Big[ \left( \sum_{i=0}^{n-1} \left[ \big(B(t_{l+1}) - B(t_l)\big)^2\right] - t \right)^2\Big] = E\left[\Big(\sum_{i=0}^{n-1} X_i \Big)^2\right] = \sum_{i=0}^{n-1}E[X_i^2] $$
Now since the 4th moment of a normally distributed random variable with variance $\sigma^2$ is $3\sigma^4$, we have
\begin{eqnarray*}
E[X_i^2] &=& E[ ( B(t_{i+1}) - B(t_i) )^4] -2(t_{i+1} - t_i)E[ (B(t_{i+1}) - B(t_i) )^2] + (t_{i+1} - t_i)^2\\
&=& 3(t_{i+1} - t_i)^2 - 2(t_{i+1} - t_i)^2 + (t_{i+1} - t_i)^2 \\
&=& 2(t_{i+1} - t_i)^2
\end{eqnarray*}
Therefore 
\begin{eqnarray*}
\sum_{i=0}^{n-1} E[X_i^2] &=& 2\sum_{i=0}^{n-1} (t_{i+1} - t_i)^2 \\
&\le&2(\Delta t_n) \sum (t_{i+1} - t_i) \\
&=& 2t\Delta t_n
\end{eqnarray*}

and so $$E\left[ \Big( \sum_{i=0}^{n-1} X_i \Big)^2 \right] \le 2t\Delta t_n$$ and by Chebyshev's inequality (see appendix) 
$$P\left(\Big| \sum_{i=0}^{n-1} X_i \Big| > \epsilon \right) \le {E\left[  \left( \sum_{i=0}^{n-1} X_i \right)^2 \right] \over \epsilon^2} \le {2t\Delta t_n \over \epsilon^2} $$

Now choose a sequence $\pi_n$ of the $2^n$ dyadic divisions of the interval $[0,1]$ with $n=1,2,\hdots$ and note that $\sum_{i=0}^\infty {1\over2^n} = 2$. Then
$$\sum_{n=1}^\infty P\left(\Big| \sum_{t_i \in \pi_n} \left[ B(t_{i+1}) - B(t_i)\right]^2 - (t_{i+1} - t_i)\Big| > \epsilon \right) \le   {2t\sum_{n=1}^\infty \Delta t_n \over \epsilon^2} < \infty$$
   
By the first Borel-Cantelli lemma (see appendix) there can only be a finite number of members of the $\pi_n$ sequence that are such that 
$$\Big| \sum_{t_i \in \pi_n}  \left[ B(t_{i+1}) - B(t_i)\right]^2 - (t_{i+1} - t_i) \Big| > \epsilon $$ Therefore we can conclude that there must exist an $N \ge 1$ such that for all $n>N$ we must have
$$\big| \sum_{t_i \in \pi_n}  \left[ B(t_{i+1}) - B(t_i)\right]^2 - (t_{i+1} - t_i) \big|  < \epsilon$$ Hence we conclude that 
$$\sum_{t_i \in \pi_n}          \left[ B(t_{i+1}) - B(t_i)\right]^2 - (t_{i+1} - t_i) \rightarrow 0$$ wp1 (a.s.) Therefore we have shown that the quadratic variation $V\goes1$ on $[0,1]$ in q.m. and a.s. 
 
\subsection{Brownian motions have infinite length}

This section also works at the sample function level. \\

Now define the Total Variation of B.M. on $[0,1]$ by $$T = \lim_{\ninfty} T_n \equiv \lim_{\ninfty} \sum_{l=1}^{2^n} \big| B({l\over 2^n}) - B({l-1\over2^n})\big|.$$

By the triangle inequality, since the (n+1)-th mesh cuts the n-th in half, $T_n$ is monotonically increasing and converges to $T$, i.e., $T_1 < T_2< \hdots < T_n < T_{n+1} < \hdots $ and $$\lim_{\ninfty} T_n \to T$$ which we designate with the $\uparrow$ symbol (see Section [\ref{sec:notation}]), viz., 
$$T_n \le T_{n+1} \uparrow T$$ So that $T$ is well-defined a.s.and is a lower bound for the path length, since $\{ B(t) \}$ is continuous a.s. \\ Now $$V_n = \sum_{l=1}^{2^n} \left[ B({l\over2^n}) - B({l-1\over2^n})\right]^2$$ so that

$$V_n \le \left[\max_{1\le l \le 2^n} \left| B({l\over2^n}) - B({l-1\over2^n}) \right|\right] \cdot \big(\sum_{l=1}^{2^n} \left| B({l\over2^n}) - B({l-1\over2^n})  \right| \big)$$ or 
$$V_n \le M_n\cdot T_n$$where $M_n = \left[\max_{1\le l \le 2^n} \left| B({l\over2^n}) - B({l-1\over2^n}) \right|\right]$. Then $$T \ge T_n \ge {V_n\over M_n}.$$ As $\ninfty, M_n \rightarrow 0$ (a.s.) by B.M. path continuity. Since $V_n \rightarrow 1$ (a.s.) it follows that $T = \infty$ (a.s.)\\

\subsection{On the Non-Differentiability of B.M. paths}
Define, for $\{Z_i\}$ IID $N(0,1)$, \{ $r_i(x), 0\le x \le 1$ orthonormal (o.n.) functions\}

$$W_n(t) = \sum_{i=1}^n Z_i \int_0^t\, r_i(x) \, dx,$$ Then $W_n(t) \rightarrow B(t)$ in quadratic mean (q.m.). If $B(t)$ had differentiable paths, one would suppose that the derivatives $${d\over dt} W_n(t) = \sum_{i=1}^n\, Z_i r_i(t)$$ would be a Cauchy sequence in q.m., that is that $$E\big[\left({d\over dt}W_n(t) {d\over dt}W_n(t) \right)^2\big] \rightarrow 0$$ as $\ninfty$. But $$\hbox{Var}\big[{d\over dt} W_n(t)\big] = \sum_{i=1}^n r_i^2(t) \uparrow \infty,$$ so that $$\{ {d\over dt} W_n(t)\}$$ cannot converge in q.m.

\subsection{Brownian motions are ``continuous''}

Define a process $K(t)$ to be a Normal Process with $0$ mean $E[K(t)] = 0$ and with covariance function given by 
\begin{eqnarray*}
\hbox{Cov}[K(s), K(t)] &=& \sigma^2 \delta(|t-s|)\\
\hbox{Var}[K(t)]           &=& \infty \\
\end{eqnarray*} where $\delta(x)$ is the Dirac-delta function. \\

Claim $$W(t) = \int_0^t\, K(v) dv,$$ is B.M. with parameter $\sigma^2$.\\

Proof: $$E[W(t)] = \int_0^t\, E[K(v)] \, dv = 0$$ and for $s<t$ 
\begin{eqnarray*}
\hbox{Cov}[W(s), W(t)] &=& E[\int_0^s \, K(u)\, du \int_0^t \, K(v)\, dv] \\
&=& \int_0^s \int_0^t\, E[K(u)K(v)] du dv \\
&=& \sigma^2 \int_0^s \int_0^t \delta(|u-v|) du dv \\
&=& \sigma^2 \int_0^s \left[\int_0^s \delta(|u-v|) + \int_s^t \delta(|u - v|) \right]\\
&=& \sigma^2 \int_0^s \int_0^s \delta(|u-v|) = \sigma^2 \int_0^s dv = \sigma^2 s\\
\end{eqnarray*}

So we see that $W(t)$ has the same properties as B.M. and hence B.M. can be written as an ``integral'' $$B(t) = \int_0^t \, K(v)\, dv.$$ Since integration ``improves'' continuity properties we can conclude that  ``$B(t)$ is continuous''.\\

However, as shown above ${d\over dt} B(t) = K(t)$ ``does not exist.'' because B.M. is non-differentiable. \\

Note: The stochastic process $K(v)$ used in the above derivation is known as ``white noise''.
\section{Ito's Isometry and Ito's Lemma}

In this section we will investigate the existence, in a certain sense, of $$``\int_0^t f(s,\omega) dB(s,\omega)''$$ where $B(t, \omega)$ is a 1-dimensional Brownian motion starting at the origin, for a class of functions $f: [0,\infty] \times \Omega \rightarrow R$.\\

Suppose $0\le S \le T$ and $f(t,\omega)$ is given, We want to define $$\int_S^T f(s,\omega) dB(s,\omega).$$ To do so we will develop the integral for a simple class of functions $f$ and then extend to more general functions by some approximation method. Let us first assume that $f$ has the form

$$\phi(t,\omega) = \sum_{j\ge0} e_j(\omega)\cdot \chi_{(j\cdot2^{-n}, (j+1)2^{-n}]} (t),$$ where $\chi_{[t_j, t_{j+1} ) } (t)$ is the Indicator Function which is equal to 1 if $t\in [t_j, t_{j+1})$ and 0 if $t$ is outside that interval and $n$ is a natural number. For such functions it is reasonable to define $$\int_S^T f(s,\omega) dB_s(\omega) = \sum_{j\ge0}[B(t_{j+1}) - B(t_j)](\omega),$$ where 

$$t_k = t_k^{(n)} =
\Bigg\{ 
\begin{array}{l}
k\cdot2^{-n}\quad \hbox{if} \quad S \le k\cdot 2^{-n} \le T\\
S \quad \hbox{if} \quad k\cdot2^{-n} < S \\
T \quad \hbox{if} \quad k\cdot2^{-n} > T
\end{array}
\Bigg\}$$

Without any further assumptions on the functions $e_j(\omega)$ the above definition leads to difficulties as the next example shows.
\begin{example}
Choose
\begin{eqnarray*}
\phi_1(t,\omega) &=& \sum_{j\ge0} B(j\cdot2^{-n},\omega)\cdot \chi_{(j\cdot2^{-n}, (j+1)\cdot2^{-n}]} (t) \\
\phi_2(t,\omega) &=& \sum_{j\ge0} B((j+1)\cdot2^{-n},\omega)\cdot \chi_{(j\cdot2^{-n}, (j+1)\cdot2^{-n}]} (t) \\
\end{eqnarray*}
Then $$E[\int_S^T \phi_1(t,\omega) dB(t,\omega)] = \sum_{j\ge0} E[B(t_j)(B(t_{j+1}) - B(t_j))] = 0,$$ since $B(t,\omega)$ has independent increments. But
\begin{eqnarray*}
E[\int_S^T \phi_2(t,\omega) dB(t,\omega)] &=& \sum_{j\ge0} E[B(t_{j+1})(B(t_{j+1}) - B(t_j))] \\
&=& \sum_{j\ge0} E[(B(t_{j+1}) - B(t_j))^2]  = T\\
\end{eqnarray*}
So, in spite of both $\phi_1$ and $\phi_2$ appearing to be reasonable appoximations to $f(t,\omega) = B(t,\omega)$, their integrals are not close to each other at all no matter how 
large $n$ is taken to be. 
\end{example}

It is natural to approximate a function $f(t,\omega)$ by $$\sum_jf(t^*_j,\omega)\cdot \chi_{(t_j,t_{j+1}]}(t)$$ where the points $t^*_j$ belong to the intervals $(t_j, t_{j+1}]$, and then define $\int_S^T f(t,\omega) dB(t,\omega)$ as the limit of $\sum_j f(t^*_j, \omega)[B(t_{j+1}) - B(t_j)](\omega)$ as $\ninfty$. However as the example above shows that -- unlike the Riemann-Stieiltjes integral -- it does make a difference here what points $t^*_j$ we choose.\\

There are two standard choices that have turned out to be the most useful:\\

1) $t^*_j = t_j$ (the left end point), which leads to the {\elevenit Ito integral} which we will denote by $$\int_S^T f(t,\omega) dB(t,\omega),$$
2) $t^*_j = (t_j + t_{j+1})/2$ (the mid point), which leads to the {\elevenit Stratonovich integral} denoted by 
$$\int_S^T f(t,\omega) \circ dB(t,\omega),$$

We will work with square-integrable functions whose mean-square is assumed to be finite, i.e., $$E[\int_S^T f(t,\omega)^2\,dt] < \infty.$$ We will also work with ``elementary functions'' of the form $$ \phi(t,\omega) = \sum_j\,e_j(\omega) \cdot \chi_{[t_j, t_{j+1})}(t) $$ where $e_j(\omega)$ are a set of $n$ real numbers indexed by $j = 1, 2, ..., n$. $\phi(t,\omega)$ is a ``simple function'' and can used to approximate an arbitrary measurable function on $(T,\Omega)$.\\

Ito's Isometry follows. An {\elevenit Isometry}\/ is a transformation that preserves the norm -- in this case the $L_2$-norm with respect to square-integrable functions. Another famous isometry is the Plancherel-Parseval theorem (see Eq. [\ref{eqn:Plancherel}]).

\begin{lemma}{The Ito isometry:}
If $\phi(t,\omega)$ is bounded and elementary then 
$$E\Big[\Big( \int_S^T\, \phi(t,\omega)\, dB_t(\omega)\Big)^2\Big] = E\Big[\int_S^T\, \big(\phi(t,\omega)\big)^2 \, dt\Big].$$ 
\end{lemma}

\begin{proof}
Put $\Delta B_j = B(t_{j+1}) - B(t_j)$. Then 
$$E[e_ie_j \Delta B_i \Delta B_j] = \Bigg\{
\begin{array}{l}
0 \quad \hbox{if} \quad i\neq j \\
E[e_j^2] \quad \hbox{if} \quad i=j \\
\end{array}$$ using the fact that $e_i e_j\Delta B_i$ and $\Delta B_j$ are independent if $i<j$.
Therefore 
\begin{eqnarray*}
E[(\int_S^T \phi\, dB)^2] &=& \sum_{i,j} E[e_ie_j\Delta B_i\Delta B_j] = \sum_{j} E[e_j^2]\cdot (t_{j+1} - t_j) \\
&=& E[\int_S^T \phi^2 dt].\qed
\end{eqnarray*}
\end{proof}

It is possible to extend the application of the Ito isometry to functions, $f$ which can be represented by the limit as $\ninfty$ of a sequence of elementary functions $\phi_n$ in the sense of mean squares, i.e., 
$$E[\int_S^T (f- \phi_n)^2 dt] \rightarrow 0 \hbox{~as~} \ninfty.$$

In this case we would define the Ito integral $\cal I$ as $${\cal I}[f](\omega) := \lim_{\ninfty}\int_S^T \phi_n(t,\omega)\, dB(t,\omega).$$ In this case the Ito isometry becomes
$$E\Big[\Big( \int_S^T\, f(t,\omega)\, dB_t(\omega)\Big)^2\Big] = E\Big[\int_S^T\, f(t,\omega)^2 \, dt\Big].$$ 

Let us know look at Ito's lemma:
\begin{lemma}{Ito's lemma}
$$\int_0^t B(s,\omega)\, dB(s,\omega) = {1\over2}B^2(t, \omega) - {1\over2}t.$$
\end{lemma}

Before going into the proof the Ito's lemma consider the values of the Brownian motion (a continuous function) at discrete times $t \in \{t_1, t_2, ..., t_n \}$ or the set $\{B_1, B_2, ..., B_n\}$.\\

Let $\Delta B_j = B_j - B_{j-1}$ and let $\Delta B^2_j = B^2_j - B^2_{j-1}$. Then $$(\Delta B_j)^2 = B^2_j + B^2_{j-1} - 2B_{j-1}B_j$$ and 
\begin{eqnarray*}
(\Delta B_j)^2 + 2B_j\Delta B_j &=& B^2_j + B^2_{j-1} - 2B_{j-1}B_j \\
& = & B^2_j + B^2_{j-1}  - 2B_{j-1}B_j + 2B_{j-1}B_j  - 2B^2_{j-1} \\
& = & B^2_j - B^2_{j-1} \\
& = & \Delta B^2_j
\end{eqnarray*}

or $$ \Delta B^2_j = 2B_{j-1}\Delta B_j + (\Delta B_j)^2$$

Or equivalently, 
$$B_{j-1}\Delta B_j - {1\over2}\Delta B^2_j  + {1\over2}(\Delta B_j)^2 = 0$$

Taking ``infinitesmal limits'' we can write $$\Delta B^2 = 2\,B\, \Delta B + (\Delta B)^2$$

Now Ito's lemma can be shown to hold ``on average'' by taking the expectation of the above result. Because Brownian motions have properties of normal distributions, then this gives

\begin{eqnarray*}
E[\Delta B^2] &=& E[2\,B\, \Delta B + (\Delta B)^2]\\
&=& 2B\, \Delta B + \sigma^2 \Delta t
\end{eqnarray*}
where $\sigma^2$ is the standard deviation associated with the Brownian motion. This is Ito's lemma. In Integral Form is can be written as 

$$\int B\,dB = {1\over2}B^2 - {1\over2} \sigma^2 t$$

Let us know prove Ito's lemma in detail
\begin{proof}
Put $\phi_n(s,\omega) = \sum B_j(\omega) \cdot \chi_{[t_j, t_{j+1})}(s)$, where $B_j = B(t_j, \omega)$ and $B_s = B(s, \omega)$. Then 
\begin{eqnarray*}
E[\int_0^t(\phi_n - B(s,\omega) )^2 ds] &=& E[\sum_j \int_{t_j}^{t_{j+1}} (B_j - B_s)^2 \, ds]  \\
= \sum_j\int_{t_j}^{t_{j+1}} (s-t_j) ds  &=& \sum_j {1\over2}(t_{j+1} - t_j)^2 \rightarrow 0 \hbox{~as~} \Delta t \rightarrow 0. 
\end{eqnarray*}
Therefore $$\int_0^t B_s\, dB_s = \lim_{\Delta t_j\rightarrow 0} \int_0^t \phi_n\, dB_s.$$
Now
\begin{eqnarray*}
&\null& E[( {1\over2}B_t^2 - {1\over2}t - \sum_j B_j \Delta B_j)^2] \\
&=&E[{1\over4} B_t^4 + {1\over4}t^2 + (\sum_j B_j \Delta B_j)^2 - B_t^2\cdot \sum_j B_j\Delta B_j - {1\over2}tB_t^2 + t\cdot \sum_j B_j\Delta B_j] \\
&=& {3\over4}t^2 + {1\over4}t^2 + E[(\sum B_j \Delta B_j)^2] - E[B_t^2\cdot \sum B_j\Delta B_j] - {1\over2}t^2 + 0. 
\end{eqnarray*}
Also, $$E[(\sum B_j \Delta B_j)^2] = \sum_{i,j} E[B_iB_j\Delta B_i \Delta B_j] = \sum_iE[B_i^2(\Delta B_i)^2]  \sum_i t_i \Delta t_i$$ and
\begin{eqnarray*}
&\null&E[B_t^2\cdot \sum B_j \Delta B_j] = \sum_jE[B_t^2B_j\Delta B_j] = \sum_j E[\{(B_t - B_{j+1}) + \Delta B_j + B_j\}^2\cdot B_j\cdot \Delta B_j]\\
&=&\sum_jE[(B_t - B_{j+1})^2 B_j\Delta B_j] + \sum_j E[B_j\cdot (\Delta B_j)^3] + \sum_jE[B_j^3\cdot \Delta B_j] \\
&+& 2\sum E[(B_t - B_{j+1}) B_j\Delta B^2_j] + 2\cdot\sum_j E[(B_t-B_{j+1})B_j^2\Delta B_j + 2 \sum_j E[B_j^2\Delta B_j^2]\\
&=& 0 + 0 + 0 + 0 + 0 + 2\sum_j t_j\cdot \Delta t_j
\end{eqnarray*}

Hence we have Ito's lemma: $$E[({1\over2}B_t^2 - {1\over2}t - \sum_j B_j \Delta B_j)^2] = {1\over2}t^2 - \sum_j t_j\Delta t_j \rightarrow 0 \hbox{~as~} \Delta t_j \rightarrow 0\qed$$
\end{proof}

This is the end of the demonstration of Ito's lemma.\\

As is well-known, Langevin's description of Brownian motion introduces a physical  modification to take into account viscosity (friction) \cite{Sjogren}, \cite{CalTech}. The result is that physical Brownian motion has finite total variation and the particles undergoing Brownian motion do not 
travel at infinite speed.\\

 It would be interesting to explore the effects of special relativity which require that the speed of any particle is less than the speed of light (relativistic Brownian Motion).  Also, in Quantum Field Theory -- the commutators of physical observables vanish outside the light-cone, however the propagators associated with quantum mechanical wave-functions have exponentially decaying tails outside the light cone. We talk about time-like and space-like particles to distinguish particles which have time-like or space-like momentum 4-vectors, or equivalently are propagating inside or outside the light cone. The paradox is that even though it is possible for a propagator associated with a physical particle not to vanish outside the light cone, nevertheless there is a strict requirement of microscopic causality that the commutators associated with physical observables must vanish identically outside the light cone. A study of relativistic Brownian motion may shed some light on this subject.
\section{Martingales}
\subsection{Martingales}
The standard reference on martingales is \cite{Doob}. A stochastic process $\{X(t), t \ge 0\}$ with finite means is said to be a {\elevenit continuous parameter martingale}\/ if for any set of times 
\be t_1 < t_2 < \hdots < t_n < t_{n+1},\label{eqn:timeSet}\ee 
\be E[X(t_{n+1}) | X(t_1), \hdots, X(t_n) ] = X(t_n).\ee 
Therefore the conditional expectation of $X(t_{n+1})$, given the history of values $X(t_1), \hdots, X(t_n)$ is equal to the most recently observed value $X(t_n)$. 
Consider a stochastic process $\{X(t), t \ge 0\}$ with independent increments and finite means together with the above set of times in Eq. [\ref{eqn:timeSet}]. 
Note that $$X(t_{n+1}) = X(t_n) + [X(t_{n+1}) - X(t_n)].$$ 
Because $X(t)$ has independent increments, then $[X(t_{n+1}) - X(t_n)]$  is independent of any elements in the set 
$\{ X(t_1), \hdots, X(t_n) \}$ because those can be considered boundary points of the respective intervals $[X(t_i) - X(0) ]$ for each $i$ (e.g., assuming $X(0) = 0$). 
Therefore \be E[ X(t_{n+1}) - X(t_n) | X(t_1), \hdots, X(t_n) ] = E[ X(t_{n+1}) -  X(t_n)]  = m(t_{n+1}) - m(t_n) \ee where $m(t) = E[X(t)] < \infty$ by assumption.  
Putting these facts together we see that \be E[X(t_{n+1}) | X(t_1), \hdots, X(t_n) ] = X(t_n) + m(t_{n+1}) - m(t_n)\ee 
If $m(t_{n+1}) = m(t_n)$, then 
\be E[X(t_{n+1}) | X(t_1), \hdots, X(t_n) ] = X(t_n) \ee  and the stochastic process is a martingale. \\

A stochastic process $\{X_n, n = 1, 2 \hdots, \}$ with finite means is said to be a {\elevenit discrete parameter martingale}\/ if for any integer $n$
\be E[X_{n+1} | X_1, \hdots, X_n ] = X_n.\ee

Let us consider some example martingales:

\begin{example}{\elevenit Sums of independent, zero mean, random variables}\/
Consider the consecutive sums of independent random variables with zero mean 
\be S_n = \sum_{i=1}^n X(t_i),\quad\quad E[X(t_i)] = 0 \ee
Then $S_n$ is a martingale because
$S_{n+1} = X_{n+1} + S_n$ and therefore 
\bearray E[S_{n+1} | S_1, \hdots, S_n ] &=& E[S_{n+1} | X(t_1), \hdots, X(t_n) ] = E[X(t_{n+1}) + S_n | X(t_1), \hdots, X(t_n)] \\
&=&E[X(t_{n+1}) | X(t_1), \hdots, X(t_n)] + E[S_n | X(t_1), \hdots, X(t_n)]\\ 
&=&  E[X(t_{n+1})] + E[S_n | X(t_1), \hdots, X(t_n)]  = 0 + S_n \\
&=& S_n 
\eearray since $S_n$ is fixed if $X(t_1), \hdots, X(t_{n+1})$ are given.\demo\end{example}

\begin{example}{$Y_n = (S^2_n - \sigma^2n)$ \elevenit is a martingale.}
Consider again the sums of independent, zero mean random variables with $E[X^2_i] = \sigma^2$ and square the sums and form the difference  $S^2_n - \sigma^2n $. Then take the conditional expectation to obtain
\bearray
 E[ (S^2_n - \sigma^2 n) | X_1, \hdots, X_{n-1}] &=& E[ (S_{n-1} + X_n)^2 - \sigma^2 n | X_1, \hdots X_{n-1}] \\
 &=& E[S^2_{n-1} | X_1, \hdots X_{n-1}] + E[X^2_n | X_1, \hdots X_{n-1}] + 2E[X_n]E[S_{n-1} | X_1, \hdots, X_{n-1}] - \sigma^2 n  \\
 &=& S^2_{n-1} + E[X^2_n] - \sigma^2 n  \\
 &=& S^2_{n-1} + \sigma^2 - \sigma^2 n  \\
 &=& S^2_{n-1} - \sigma^2(n - 1) \eearray
Since $Y_{i}$ is determined completely by $X_1, \hdots, X_{i},\quad i = 1, 2, \hdots, n-1$, we have by the above that 
$$E[Y_n | Y_1, \hdots Y_{n-1} ] = Y_{n-1}$$ and therefore $Y_n = (S^2_n - \sigma^2 n)$ is a martingale.\demo\end{example}

One can also conclude that the unconditional expectations will form a chain such that 
$$ E[S^2_n - \sigma^2n] = E[S^2_{n-1} - \sigma^2(n-1)] = \hdots = E[S^1 - \sigma^2] = E[X^2_1] - 1 = 0 $$ and therefore that
\be E[S^2_n] = \sigma^2 n. \ee

\begin{example}{\elevenit Brownian motion is a martingale} For $s<t$ we have
\bearray E[B(t) | B(s) ] &=& E[B(t) - B(s) + B(s) | B(s) ] \\
&=& E[B(t) - B(s) | B(s)] + E[B(s)|B(s)] \\
&=& 0 + B(s) \quad\hbox{by independent increments} \\
&=& B(s) 
\eearray Therefore Brownian motion satisfies the martingale condition $E[B(t) | B(s) ] = B(s)$ for $s < t$.\demo\end{example}
Let us compute the conditional expectation for exponentiated Brownian motion.
Assuming $s <  t$
\bearray
 E\big[\exp[\theta B(t) ] | B(s)\big]  &=& E[\exp\big(\theta[B(t) - B(s)] + \theta B(s) \big) | B(s)] \\
&=& \exp[\theta B(s)] E[\exp\big(\theta[ B(t) - B(s)] \big)] 
\eearray by independent increments. Since $B(t) - B(s)$ is normally distributed with mean $0$ and variance $\sigma^2 (t-s)$ we can make use of the m.g.f. for the normal distribution to compute 
\be E\big[\exp\big(\theta[ B(t) - B(s)] \big)\big] = \exp[\theta^2 \sigma^2 (t-s)/2] \ee and arrive at
\be  E\big[e^{\theta B(t)} | B(s)\big]  = e^{\theta B(s) + {1\over2}\theta^2\sigma^2(t-s)} \label{eqn:expBrownian}\ee
So, there is an extra term on the RHS of Eq. [\ref{eqn:expBrownian}] which means that $\exp[\theta B(t)]$ is not a martingale. However, as this can be rectified as shown in the following. 
 
\begin{example}{\elevenit Exponentiated Brownian Motion Martingale}
Define a stochastic process $$X(t,\theta) = e^{\theta B(t) - {1\over2}\theta^2 \sigma^2 t}$$
In this case $X(t,\theta)$ will be a martingale because
\bearray
 E[X(t,\theta) | X(s,\theta)] &=& E[X(t,\theta) | B(s), \sigma^2(t-s) ] \\
&=&  e^{\theta B(s) + {1\over2}\theta^2\sigma^2 (t-s) - {1\over2}\theta^2\sigma^2 t } \\
&=&   e^{\theta B(s) - {1\over2}\theta^2\sigma^2 s} \\
&=& X(s,\theta)
\eearray \demo\end{example}
The above martingale has the interesting property that it can be used to {\elevenit generate}\/ more martingales by differentiating $X(t,\theta)$ with respect to $\theta$ and setting $\theta = 0$. In other words, there is a whole series of martingales which are related to Brownian motion by using $X(t,\theta) = e^{\theta B(t) - {1\over2}\theta^2 \sigma^2 t}$ as a generating function. \\

Suppose we are observing a 1-dimensional Brownian motion, $B(t)$ that begins at $B(0) = 0$. Given two real numbers $a$ and $b$ we ask what is the probability, $p$, that
the Brownian motion hits $b$ before it hits $-a$? ( $b$ is to the right of 0 and $-a$ is to the left). Let $T$ be the time to hit $b$ or $-a$ whichever is first. $T$ is an example of 
a {\elevenit stopping time}. One needs to show that $T$ is finite -- in other words, it must be shown that the Brownian motion process does not stay between $-a$ and $b$ forever. We assume that this is true and that the Brownian motion eventually hits $-a$ or $b$.
The probability $p$ is given by 
\be p = P[\hbox{hit $+b$ before $-a$}] = P[B(T) = b] \ee  
Solution: $B(t)$ is a martingale. Therefore
$E[B(t) | B(s) ] =  B(s)$ in general and for $t = T$ we have 
$$ E[B(T) | B(0) ] = B(0) = 0 $$
Now since $B(T) = b$ with probability $p$, then $B(T) = -a$ with probability $1-p$ and so 
\be E[B(T) | B(0) ] = bp + (-a)(1-p) = B(0) = 0 \ee or
\bearray
 0 &=&  bp -a + ap\\
a  &=& p(a+b) \\
p &=& {a\over a + b}
\eearray 
Therefore the probability that the Brownian motion will first hit $b$ is $a/(a+b)$. \\

Let us further seek to find the expected time $E[T]$ when the Brownian motion will first encounter either $b$ or $-a$. Now Brownian motion satisfies
$E[B^2(t)] = \sigma^2 t$ and therefore $E[B^2(t) | t = T] = \sigma^2T$ and so \be E\big[E[B^2(t) | t = T] \big] = \sigma^2E[T] \ee

Now, based on the probabilities we found above, we have
\be B^2(T) = \Bigg\{ 
\begin{array}{rcl}
b^2 & \hbox{probability} & p \\
(-a)^2 & \hbox{probability}  & 1-p
\end{array}
\ee  

Putting everything together we find
\bearray
\sigma^2E[T] &=& E[B^2(T)] \\
&=& b^2p + a^2(1-p) \quad \hbox{and we have that $p = a/(a+b)$}\\
&=& b^2{a\over a+b} + a^2{b\over a+b}\\
&=& ab{a+b\over a+b} \\
&=& ab
\eearray
In summary,  the probability that B(t) hits $b$ before $-a$ is $p=a/(a+b)$ and the expected time to hit is $E[T] = ab/\sigma^2$. 

\subsection{Sub-and-super martingales}
A {\elevenit submartingale} is a stochastic process that satisfies the following inequality
\be E[X_k | X_1, \hdots, X_{k-1}] \ge X_{k-1} \ee

A {\elevenit supermartingale} is a stochastic process that satisfies the following inequality
\be E[X_k | X_1, \hdots, X_{k-1}] \le X_{k-1} \ee

\begin{example}
Consider again the squares of sums $S^2_k$ above. 
\bearray 
E[ S^2_k | X_1, \hdots, X_{k-1} ] &=& E[ (X_k + S_{k-1})^2 | X_1, \hdots, X_{k-1}]\\
&=& E[X_k^2 | X_1, \hdots, X_{k-1}] + E[S^2_{k-1} | X_1, \hdots, X_{k-1}] + 2E[X_k] E[S_{k-1} | X_1, \hdots, X_{k-1}]\\
&\ge& E[S^2_{k-1} | X_1, \hdots, X_{k-1}]  = S^2_{k-1}
\eearray because $E[X_k] = 0.$Therefore $S^2_k$ is a submartingale because $E[S^2_{k} | X_1, \hdots, X_{k-1}] \ge S^2_{k-1}$.\demo\end{example}


%Future plans include Wiener Filters, Kalman-Bucy Filters and Stochastic Differential Equations. 
%\section{Statistics}

The main random variable we will concern ourselves with is the Gaussian or normal random variable with mean $\mu$ and variance $\sigma^2$, denoted by $N(\mu, \sigma^2)$. 
The Standard Normal Distribution has parameters $\mu =0$ and $\sigma = 1$ and is denoted by $N(0,1)$. If $X$ is a normal random variable distributed according to $N(\mu,\sigma^2)$ we write 
$X \sim N(\mu, \sigma^2)$
The Gaussian random variable is continuous and can take on any value in the real number line. The probability distribution function for a $X \sim N(\mu,\sigma^2)$ random variable is given by 
$$f_X(x) = {1\over \sqrt{2\pi}\sigma}\, e^{-(x-\mu)^2/2\sigma^2} \quad\quad -\infty < x < \infty$$
The probability for $X$ to take on values in a subset $B \subset R$ is given by $$P\{X \in B\} = \int_{x \in B}\, f_X(x)\, dx.$$

The probability density function is normalized to unity. Using the change-of-variable $u = (x-\mu)/\sigma$: 
\begin{equation}
\int_{-\infty}^{\infty} \, f_X(x) dx = {1\over \sqrt{2\pi}} \int_{-\infty}^{\infty}\, e^{-u^2/2}\, du = 1
\label{eqn:gaussnormalization} \end{equation}\\

If $X \sim N(\mu, \sigma^2)$, then $Z = \alpha X + \beta$ will be distributed according to $Z \sim N(\alpha\mu + \beta, \alpha^2\sigma^2)$: 
\begin{proof}
\begin{eqnarray*}\large
F_Z(a)    &=& P(Z\le a)\\
               &=& P(\alpha X + \beta \le a) \\
               &=& P\left(X \le {a - \beta\over \alpha}\right) \\
               &=& F_X\left({a - \beta\over \alpha}\right) \\ \\
               &=& {1\over \sqrt{2\pi}\sigma} \int_{-\infty}^{\left({a-\beta\over \alpha}\right)} \, e^{-(x-\mu)^2/2\sigma^2}\, dx\\ \\
               &=& {1\over \sqrt{2\pi}} \int_{-\infty}^{a}\, e^{-{ [z - (\alpha\mu + \beta) ]^2\over2\alpha^2\sigma^2} } \, dz
\end{eqnarray*}
As shown above, $Z \sim N(\alpha\mu + \beta, \alpha^2\sigma^2)$.\qed
\end{proof}

The CDF for the standard normal random variable is $\Phi(a)$ and is given by 
\be \Phi(a) = {1\over \sqrt{2\pi}} \int_{-\infty}^{a}\, e^{-{ z^2/2} } \, dz \label{eqn:Phi}\ee

%\section{Applications in Estimation and Filtering}
\subsection{Linear Models}

\clearpage

\section{Appendices}

\subsection{Abbreviations and Notation}
\label{sec:notation}

$\define$ \quad  ~is defined as \\ 
IID \quad Independent Identically Distributed \\
SII \quad Stationary Independent Increments \\  
{\elevenit iff} \quad ~if and only if \\ 
a.s.  \quad almost surely\\
i.o. \quad infinitely often\\
m.s. \quad mean square (same as quadratic mean)\\
o.n. \quad orthonormal\\ 
q.m. \quad  quadratic mean (same as mean square)\\
wp1 \quad with probability 1\\
w.r.t \quad with respect to \\
$\{x_n\}$\quad ~ is a list or sequence of numbers $x_n, n = 1, 2, \hdots$ \\
$x_n \to x$ ~means $\displaystyle{\lim_{\ninfty} x_n = x}$ \\
$\uparrow$\quad is used to denote monotonically increasing sequence $\{x_n\}$, which converges from below:  $$x_n \uparrow x \implies x_1 < x_2 < \hdots \hbox{~and~} \lim_{\ninfty} x_n = x.$$
For sets $A_n \uparrow A$ means $A_1 \subset A_2 \subset \hdots $ and $\displaystyle{A=\bigcup\limits_{n=1}^\infty A_n}$.\\

$\downarrow$\quad is used to denote a monotonically decreasing series which converges from above: $$x_n \downarrow x \implies x_1 > x_2 > \hdots \hbox{~and~} \lim_{\ninfty} x_n = x.$$
For sets $A_n \downarrow A$ means $A_1 \supset A_2 \supset \hdots $ and $\displaystyle{A=\bigcap\limits_{n=1}^\infty A_n}$.\\

%Definition of Ergodic Stochastic Processes:\\
%1) Processes such that the {\elevenit time average}\/ equals the {\elevenit assemble average} are called {\elevenit ergodic}. \\
%2) Markov chains: positive, recurrent, aperiodic states are called {\elevenit ergodic}.\\
%3) Markov chains: if the limiting probability for finding the system in state $j$ is independent of the initial state $i$, then the state $j$ is called {\elevenit ergodic}. If all the states are ergodic, then 
%the Markov chain is called {\elevenit ergodic}. An aperiodic positive-recurrent Markov chain is defined to be an ergodic chain, and the interpretation of this is the same as with finite-state Markov chains. In particular, for ergodic chains, the steady-state probability $\pi_j$ is, with probability 1, the time average rate of that state. It is also the time and ensemble average rate of the state, and, from 
%$$\lim_{\ninfty} P^n_{ij} = {1\over\overline{T}_{jj}},$$ the limiting probability of the state, independent of the starting state.

\subsection{Equivalence Relations}
\label{sec:equivalence}

An important technique for dividing a set into mutually exclusive subsets is the method of equivalence relations \cite{Hall}.
\begin{definition}
Suppose in a set $A$ we have a relation $R$ defined between pairs of elements $x$ and $y$, $x R y$ indicating that $x$ and $y$ stand in the relation $R$. 
Suppose that $R$ has the following three properties: \\

(1) $R$ is reflexive: i.e., $x R x$ for all $x\in A$.\\
(2) $R$ is symmetric: i.e., $x R y \implies y R x$.\\
(3) $R$ is transitive: i.e., $x R y$ and $y R z \implies x R z$.
Then $R$ is an {\elevenit Equivalence Relation}.
\end{definition}
\begin{theorem}
An equivalence relation $R$ divides $A$ into mutually exclusive sets so that every element of $A$ is in one and only one subset, and so that two elements are in the same 
subset iff they stand in the the relation $R$ to one another. 
\end{theorem}
\begin{proof}
Given any element $x$ in $A$ consider all the elements $y$ such that $x R y$. These elements form a subset of $A$ which we call $A_x$. Now two elements are in the same subset $A_x$, iff they stand 
in the relation $R$ with one another. For suppose $y R z$ and $y \in A_x$. Then $x R y$ and, because $R$ is transitive, $x R z$. Then $z \in A_x$. Also, if $y$ and $z$ are both in $A_x$, then $x R y$ and 
$x R z$. By the symmetric property $y R x$. Together with $x R z$ we have $y R x$ and $x R z$ and therefore $y R z$ by transitivity.\\

For each element $x \in A$ we have a subset $A_x$. These subsets will not all be distinct. We next show that any two such subsets are either mutually exclusive of identical. Suppose both $A_x$ and $A_y$ both have an element $z$ in common. Then $x R z$ and $y R z$,. By the symmetric property we have $z R y$ and therefore $x R y$ by transitivity. Now pick any element $w$ of $A_x$, $x R w$. Because 
$x R y$ we also have $y R x$ and therefore $y R w$ also by transitivity. Therefore $w \in A_y$, or $A_x \subset A_y$. By the same reasoning, exchanging $A_x$ and $A_y$ and going through the same line of reasoning, we must also have $A_y \subset A_x$. So $A_x$ and $A_y$ are identical. Therefore we have a list of mutually exclusive subsets $A_{x_1}, A_{x_2}, \hdots$. Also, for any element $z \in A$ we have $z R z$ by the reflexive property, $z$ must be in one of the subsets, i.e., $A_z$. \qed
\end{proof}

\subsection{Various Modes of Convergence}
\label{sec:convergence}
Given random variables $Z, Z_1, Z_2 ...$ we say that \\

i) the sequence $\{Z_n\}$ converges to $Z$ {\elevenit surely}\/ if, given any $\epsilon > 0$ there exists an integer $N > 0$ such that
$$ | Z_n - Z | < \epsilon\quad\hbox{for all $n\ge N$}$$ This mode of convergence is also denoted by  $Z_n \rightarrow Z.$ \\

ii) the sequence $\{Z_n\}$ converges to $Z$ {\elevenit with probability one}\/ (wp1) iff for every $\epsilon>0$
$$\lim_{N \to \infty} P\Big[\Big(\max_{n\ge N} |Z_n - Z| \Big) > \epsilon \Big] = 0.$$ Such a sequence is said to converge {\elevenit almost surely}\/ (a.s.) which is also denoted by     
$$P[\lim_{\ninfty} Z_n = Z] = 1.$$
iii) the sequence $\{Z_n\}$ converges to $Z$ {\elevenit in probability}\/ if for every $\epsilon>0$ $$\lim_{\ninfty} P\big[ |Z_n - Z | > \epsilon \big] = 0,$$ 
iv) the sequence $\{Z_n\}$ converges to $Z$ in {\elevenit mean square}\/ (m.s.)  or in {\elevenit quadratic mean}\/ (q.m.), if each random variable $Z_n$ has a finite mean square and if 
$$\lim_{\ninfty}\, E\big[ (Z_n - Z)^2 \big] = 0.$$
v) a sequence of random variables can also converge {\elevenit in distribution}\/ if their probability distribution functions converge to a limiting function $$\lim_{\ninfty} f_{Z_n}(z) = f_Z(z)$$everywhere except on a set of measure zero.\\

vi) In addition to the above convergence types a sequence $\{ R(v), v=0,1, \hdots \}$ is said to converge to 0 in Ces\`aro mean as $v \to \infty$ if $$\lim_{t\rightarrow\infty} {1\over t}\sum_{v=0}^{t-1}\,R(v) = 0.$$

\subsection{DeMorgan's Laws} The following useful relationship between the three basic operations of forming unions, intersections, and complements of sets is known as {\elevenit DeMorgan's laws}\/
\label{sec:DeMorgan}
\begin{eqnarray*}
\Big(\bigcup\limits_{i=1}^n A_i  \Big)^c   =  \bigcap\limits_{i=1}^n\, A_i^c \\
\Big(\bigcap\limits_{i=1}^n A_i  \Big)^c   =  \bigcup\limits_{i=1}^n\, A_i^c
\end{eqnarray*} This is a standard result in set theory. The usual method of proof is the following:
Suppose that $x \in \big(\bigcup\limits_{i=1}^n A_i  \big)^c$ then $x$ is not contained in  $\bigcup\limits_{i=1}^n A_i$, which means it is not contained in any of the events $A_i, i = 1,2, \hdots n$, implying that $x$ is contained in $A_i^c$ for all $i=1,2,\hdots n$ and therefore contained in  $\bigcap\limits_{i=1}^n\, A_i^c$. Going in the other direction, suppose $x\in  \bigcap\limits_{i=1}^n\, A_i^c$. Then $x$ is contained in $A_i^c$ for all $i=1,2,\hdots n$, which means that $x$ is not contained in $A_i$ for any $i=1,2,\hdots n$, implying that $x$ is not contained in $\bigcup\limits_{i=1}^n A_i $, which means $x \in \big(\bigcup\limits_{i=1}^n A_i  \big)^c$.\\

To prove the second DeMorgan's laws, apply the first to $A_i^c$ and use the fact that $A_i = (A_i^c)^c$ for any set. We obtain 
\begin{eqnarray*} 
\Big(\bigcup\limits_{i=1}^n A_i^c  \Big)^c  &=&  \bigcap\limits_{i=1}^n\, (A_i^c)^c \\
                                                   &=&  \bigcap\limits_{i=1}^n\, A_i 
\end{eqnarray*} By taking the complement of both sides we find the second of DeMorgan's laws:
$$\bigcup\limits_{i=1}^n\, A_i^c = \Big(\bigcap\limits_{i=1}^n A_i  \Big)^c $$ 

\subsection{Fourier Theorems}
\label{sec:Fourier}

For $n$-vectors $x$ and $k$ we define the dot product of these vectors by 
$$<x\cdot k> = \sum_{i=1}^n\, x_ik_i$$ Here $x$ is an element of $\mathbb{R}^n$ and $k$ is in the ``Fourier transform space'' or the ``momentum space'' corresponding to $\mathbb{R}^n$ which is also $n$-dimensional, but is a different space (i.e., the $k$ is an element of the dual-space of $\mathbb{R}^n$).
Let $f(x)$ be a subset of integrable and square-integrable functions on $\mathbb{R}^n$, $f(x) \subset L^1(\mathbb{R}^n) \cap L^2(\mathbb{R}^n)$. The Fourier transform of $f(x)$ is denoted by $\hat{f}(k)$ and is given in symmetric form by 
\begin{equation}\hat{f}(k)  = {1\over\sqrt{(2\pi)^n}}\int_{\mathbb{R}^n} f(x) e^{-i <x\cdot k>}\,d^nk\label{eqn:Fourier}\end{equation} The inverse Fourier transform is given also in symmetric form by 
\begin{equation}f(x)  = {1\over\sqrt{(2\pi)^n}}\int_{\mathbb{R}^n} \hat{f}(k) e^{ i <x\cdot k>}\,d^nx\label{eqn:InvFourier}\end{equation} \\
One of the main theorems of Fourier Analysis is the Plancherel-Parseval theorem which states that the Fourier transform map is an {\elevenit Isometry}\/ with respect to the $L^2$ norm. 
\begin{theorem}{Plancherel-Parseval theorem:}
\begin{equation}\int_{-\infty}^\infty  f(x)\overline{g(x)}\, d^nx  = \int_{-\infty}^\infty \hat{f}(k)\overline{\hat{g}(k)}\, d^nk,\label{eqn:Plancherel}\end{equation}
where $\overline{f}$ is the complex-conjugate of $f$
\end{theorem}

As can be seen above the functions for which the Fourier integral is applied are square-integrable functions (the are in the function space $L^2$). The norm in this space is defined by 
\be  \| f \| =  \int_{-\infty}^\infty  f(x)\overline{f(x)}\, d^nx \ee and we assume that $\| f \| < \infty$ implicitly in using Fourier integrals. Fourier series deals with periodic functions. However, no meaning can be attached to the integral 
\be f_X(\omega) = \int_{-\infty}^\infty e^{-it\omega} X(t) dt \ee
for many stochastic processes, such as covariance stationary processes, since their sample functions are non-periodic and are undamped (do not fall off at infinity in such a manner as to be in the space of $L^2$ functions). It is possible to define a generalized harmonic analysis of stochastic processes (a method of assigning to each frequency $\omega$ a measure of its contribution to the ``content'' of the process) as shown in \cite{Wiener3}.

\subsection{Cauchy-Schwarz Inequality}
\label{sec:CauchySchwarz}Here we include the proof of the Cauchy-Schwarz inequality for sequences and, by extension, to integrals:\\

(Cauchy-Schwarz inequality). Let $x=(x_1, x_2, ..., x_N)$ and $y = (y_1, y_2, ..., y_N)$ be elements of $\mathbb{R}^N$. Then 
$$ \left| \sum_{i=1}^N\, x_i y_i \right|^2 \le \sum_{i=1}^N\, x_i^2 \cdot \sum_{i=1}^N\, y_i^2$$

This inequality has many different proofs. We present the following proof. If every $x_i$ is zero, then the equality sign in the above expression holds. So let us assume that there is at least one $x_i$ not zero. We form the function $$f(\lambda) = \sum_{i=1}^N\, (y_i - \lambda x_i)^2 \ge 0$$ which is nonnegative for all real values of $\lambda$. We set 
$$A = \sum_{i=1}^N\, x_i^2 \quad  B = \sum_{i=1}^N\, x_i y_i \quad C = \sum_{i=1}^N\, y_i^2$$ Then we can expand $f(\lambda)$ as
$$f(\lambda) = A\lambda^2 - 2B\lambda + C \ge 0,~~ \hbox{for} ~~ A>0.$$ Since the only real root for $f(\lambda)$ occurs when $y_i = \lambda x_i$, then unless this is the case $f(\lambda)$ can only have complex roots. Examining the solutions to the quadratic equation we have $$\lambda = {2B \pm \sqrt{4B^2 - 4AC} \over 2A}$$ or $$\lambda = {B \pm \sqrt{B^2 - AC} \over A}.$$ $f(\lambda)$ has complex roots if $B^2 < AC$ or 
$$ \left| \sum_{i=1}^N\, x_i y_i \right|^2 \le \sum_{i=1}^N\, x_i^2 \cdot \sum_{i=1}^N\, y_i^2$$

Let us define the norm of an $n$-vector.
\begin{definition}
The vector dot-product of two $n$ vectors $u$ and $v$ is defined as 
$$u\cdot v = \sum_{i=1}^n u_i v_i.$$ 
\end{definition}

With the above definition $u\cdot u$ is the square of the magnitude of $u$. The magnitude of $u$ is also known as the norm of $u$ and is denoted by $\| u \|$.

\begin{definition}
The norm of an $n$-vector $u$ is given in terms of the vector dot-product by 
$$\| u \| = \sqrt{u \cdot u} $$
If $u$ is complex then let $\overline{u}$ stand for the complex-conjugate and define the norm of $u$ as
$$\| u \|^2 = \sum_{i=1}^n u_i \overline{u_i} = u \cdot \overline{u} $$
\end{definition}

A useful shorthand for $u \cdot u = \| u \|^2$ is $u^2$, where the squared vector is obtained by dotting the vector $u$ with itself. \\

With the above definition of norm we can also write the Cauchy-Schwarz inequality as
$$ \| x \cdot y \| \le \|x \|  \cdot \| y \|$$

The Cauchy-Schwarz inequality can be used to prove the Triangle Inequality for $n$-vectors. 
$$\|u + v \| \le \| u \| + \| v \|$$
\begin{proof}
Take $u$ and $v$ to be $n$-vectors. Then the squared norm of $u+v$ is 
$$(u+v)^2 = u^2 + 2u\cdot v + v^2$$ and by the Cauchy-Schwarz in equality $\|u\cdot v\| \le  \|u\| \|v\| $. Therefore 
$$(u+v)^2 \le u^2 + 2\|u\| \|v\| + v^2 = (\|u\| + \|v\|)^2$$ and finally, 
$$\| u + v \| \le \| u \| + \|v\|$$
\end{proof}
 
The Cauchy-Schwarz inequality can be extended to the inner product space of square-integrable complex-valued functions:

$$ \left| \int_{\mathbb{R}^N}\, f(x) \overline{g(x)}\, \right|^2 \le \int_{\mathbb{R}^N}\, |f(x)|^2\, dx \cdot \int_{\mathbb{R}^N}\, |g(x)|^2 dx$$ where $\overline{g(x)}$ is the complex conjugate of $g(x)$ and $|f(x)|^2 = f(x)\overline{f(x)}$.\\

We can also write the above inequality using the concept of expectation for random variables. \\

{\bf Cauchy-Schwarz Inequality for random variables.} For any two random variables $X$ and $Y$ we can write the Cauchy-Schwarz inequality as 
$$(E[XY])^2 \le E[X^2]E[Y^2],$$ where equality holds if and only if $X = \alpha Y$, for some constant $\alpha \in R$.

\begin{theorem}
If $E[X^2] < \infty$, then $E[|X|] < \infty$ 
\end{theorem}
\begin{proof}
Taking the square root in the Cauchy-Schwarz inequality and setting $y\equiv1$, we have
$$E[|X|] < (E[|X^2|])^{1\over2} < \infty.\qed$$
\end{proof}

It can also be shown (we skip the proof) that 
\begin{theorem}
If $E[|X|^n] < \infty$, then $E[|X|^m] < \infty$ for all $1\le m \le n$.
\end{theorem}
Hence if the $n$-th moment of $X$ is finite, then all the lower moments of $X$ will also be finite. 

\subsection{Generating Functions}
\label{sec:generating}

The Moment Generating Function  (m.g.f) $\psi_X(t)$ of the random variable $X$ is defined for all real values of $t$ by
\begin{eqnarray*}
\psi_X(t) &=& E[e^{tX}] \\ \\
&=&\Bigg\{
\begin{array}{ll}
\sum_x e^{tx} p(x) & \quad  \hbox {if $X$ is discrete with mass function $p(x)$} \\ \\
\int_{-\infty}^{\infty}\, e^{tx} f_X(x) \, dx & \quad \hbox{if $X$ is continuous with density $f_X(x)$} \\
\end{array}
\end{eqnarray*}

$\psi_X(t)$ is called the m.g.f. because all the moments of $X$ can be obtained by differentiating $\psi_X(t)$ a sufficient number of times and setting $t=0$ afterwards. For example 

\begin{eqnarray*}
\psi_X'(t) &=& {d\over dt} E[e^{tX}] \\
&=& E[{d\over dt} (e^{tX})] \\
&=& E[X e^{tX}]
\end{eqnarray*} 

Therefore $E[X] = \psi'(0)$. Similarly all the higher moments can be generated by successive differentiation.\\
 
Consider the m.g.f. for the standard Gaussian random variable $Z \in N(0,1)$:
\begin{eqnarray*}
\psi_Z(t) &=& {1\over \sqrt{2\pi}} \int_{-\infty}^{\infty}\, e^{tx}\, e^{-x^2/2}\, dx\\
               &=& {1\over \sqrt{2\pi}} \int_{-\infty}^{\infty}\, e^{-(x^2 - 2 tx)/2}\, dx\\
               &=& {1\over \sqrt{2\pi}} \int_{-\infty}^{\infty}\, e^{-{(x-t)^2\over2} + {t^2\over2}} \, dx\\
               &=& e^{t^2/2}\, {1\over \sqrt{2\pi}} \int_{-\infty}^{\infty}\, e^{-{(x-t)^2\over2}}\, dx\\
               &=& e^{t^2/2}\, {1\over \sqrt{2\pi}} \int_{-\infty}^{\infty}\, e^{-{y^2\over2}}\, dy \quad\hbox{substitution: $y = x-t$}\\
               &=& e^{t^2/2}
\end{eqnarray*}

Given that $Z$ is $N(0,1)$ we know that $X = \mu + \sigma Z$ will be normally distributed according to $N(\mu, \sigma^2)$. Therefore the m.g.f. for $X$ can be obtained using the above as follows:
\begin{eqnarray*}
\psi_X(t) &=& E[e^{tX}]\\
               &=& E[e^{t(\mu + \sigma Z)}]\\
               &=& e^{\mu t}E[e^{t \sigma Z}]\\
               &=& e^{\mu t}\psi_Z(\sigma t)\\
               &=& e^{\mu t}e^{(\sigma t)^2/2}\\
               &=& \exp\big\{\sigma^2 t^2/2 + \mu t\big\}
\end{eqnarray*}

Closely related to the m.g.f. is the Characteristic Function $\phi(u)$ of $X$ which is defined, for any real number $u$, by
\be \phi_X(u) = E[e^{iuX}]. \ee
It is possible that a random variable does not have a finite mean or variance or m.g.f. But a random variable always possesses a characteristic function.\\

\begin{theorem}\label{thm:mgfMultivariateGaussian}
Consider a set of jointly normally distributed random variables $X_1, \hdots, X_n$ with means $m_j$ and covariances $K_{jk}$ given. The m.g.f. for the set is given by
\be \psi_{X_1, \hdots, X_n}(u_1, \hdots, u_n) = \exp\big\{  \sum_{j=1}^n u_j m_j + {1\over2} \sum_{j,k=1}^n\, u_j K_{jk} u_k \big\} \ee
\end{theorem}
\begin{proof}
By definition the m.g.f for the jointly normal random variables is given by
\bearray E[e^{u_1X_1 + \hdots + u_nX_n}] &=& \int f_{X_1, ..., X_n}(x_1, ..., x_n)\, e^{u_1 x_1 + \hdots + u_n x_n} dx_1\hdots dx_n \hbox{~in matrix notation:}  \\
&=&\int {1\over(2\pi)^{n/2}} {1\over|K|^{1/2}} \exp\big\{ - {1\over2} ({\bf x} - {\bf m})^T {\bf K}^{-1} ({\bf x} - {\bf m} ) + {\bf u}^T \cdot {\bf x} \big\} \, dx_1 \hdots dx_n. \\
\eearray
Let us examine the exponent in the above equation. Using the change of variables ${\bf y = x-m}$ we have 
\bearray - {1\over2} ({\bf x} - {\bf m})^T {\bf K}^{-1} ({\bf x} - {\bf m} ) + {\bf u}^T \cdot {\bf x} &=& - {1\over2} {\bf y}^T {\bf K}^{-1} {\bf y}   + {\bf u}^T \cdot ({\bf y + m})\\
&=& \Big\{- {1\over2} {\bf y}^T {\bf K}^{-1} {\bf y}   + {\bf u}^T \cdot {\bf y}\Big\} +  {\bf u^T \cdot m}
\eearray
The last term on the RHS is independent of the variable of integration and can be moved outside the integral. 
The method of completing the square can be used to search for a ${\bf z}$ and $\lambda$ such that 
\be \Big\{- {1\over2} {\bf y} ^T {\bf K}^{-1} {\bf y}   + {\bf u}^T \cdot {\bf y} \Big\}  = - {1\over2} ({\bf y} - {\bf z})^T  {\bf K}^{-1} ({\bf y} - {\bf z} )  + \lambda \label{eqn:completeSquare}\ee
Now we assume that  ${\bf K}^{-1} $ (and also the inverse matrix ${\bf K} $) is symmetric and expand the RHS of Eq. [\ref{eqn:completeSquare}] to obtain:
\bearray -{1\over2} ({\bf y} - {\bf z})^T  {\bf K}^{-1} ({\bf y} - {\bf z} )  + \lambda &=&  
-{1\over2} {\bf y}^T  {\bf K}^{-1} {\bf y}
-{1\over2} {\bf y}^T  {\bf K}^{-1} {\bf z}
-{1\over2} {\bf z}^T  {\bf K}^{-1} {\bf y}
-{1\over2} {\bf z}^T  {\bf K}^{-1} {\bf z} + \lambda \\
&=& -{1\over2} {\bf y}^T  {\bf K}^{-1} {\bf y}
- {\bf z}^T  {\bf K}^{-1} {\bf y}
 -{1\over2} {\bf z}^T  {\bf K}^{-1} {\bf z} + \lambda \hbox{~ by symmetry of ${\bf K}$}\\
 \eearray
 By comparing the LHS of Eq. [\ref{eqn:completeSquare}] to the RHS of the above we find that
 \bearray
 {\bf z}^T &=& {\bf u^T K} \\
 \lambda &=& {1\over2} {\bf z}^T{\bf K}^{-1}{\bf z}  = {1\over2} {\bf u}^T {\bf K} {\bf K}^{-1} {\bf K}^T {\bf u} = {1\over2} {\bf u}^T {\bf K} {\bf u}  \hbox{~~also by symmetry of {\bf K}}
 \eearray 
Inserting these results into the expression for $E[e^{u_1X_1 + \hdots + u_nX_n}]$, we find 
\bearray E[e^{u_1X_1 + \hdots + u_nX_n}] &=& \int f_{X_1, ..., X_n}(x_1, ..., x_n)\, e^{u_1 x_1 + \hdots + u_n x_n} dx_1\hdots dx_n \hbox{~in matrix notation:}  \\
&=&\int {1\over(2\pi)^{n/2}} {1\over|K|^{1/2}} \exp\big\{ - {1\over2} ({\bf x} - {\bf m})^T {\bf K}^{-1} ({\bf x} - {\bf m} ) + {\bf u}^T \cdot {\bf x} \big\} \, dx_1 \hdots dx_n. \\
&=&\exp\{{\bf u}^T\cdot{\bf m} + {1\over2} {\bf u}^T {\bf K} {\bf u}  \} {1\over(2\pi)^{n/2}} {1\over|K|^{1/2}} \int \exp\big\{-{1\over2} ({\bf y} - {\bf z})^T {\bf K}^{-1} ({\bf y} - {\bf z} ) \big\} \, dy_1 \hdots dy_n. \\
&=&\exp\{{\bf u}^T\cdot{\bf m} + {1\over2} {\bf u}^T {\bf K} {\bf u} \},
\eearray
where
$$1 = {1\over(2\pi)^{n/2}} {1\over|K|^{1/2}} \int \exp\big\{-{1\over2} ({\bf y} - {\bf z})^T {\bf K}^{-1} ({\bf y} - {\bf z} ) \big\} \, dy_1 \hdots dy_n $$for any given vector ${\bf z}$.
Putting all the above together we find
\bearray \psi_{X_1, \hdots, X_n}(u_1, \hdots, u_n) 
&=& E[e^{u_1X_1 + \hdots + u_nX_n}] \\ 
&=& \exp\{{\bf u}^T\cdot{\bf m} + {1\over2} {\bf u}^T {\bf K} {\bf u} \} \\
&=& \exp\big\{  \sum_{j=1}^n u_j m_j + {1\over2} \sum_{j,k=1}^n\, u_j K_{jk} u_k \big\}
\qed
\eearray 
%can be written using the Cholesky decomposition as ${\bf K^{-1} = LL}^T$ for some lower triangular matrix ${\bf L}$.  Observe that the determinant of ${\bf K^{-1}}$ is given by 
%$$|K^{-1}| = |L L^T| = |L||L^T| = |L|^2 \hbox{~ since $|L| = |L^T|$ for any square matrix $L$}$$
%We then consider a change of variables of the form ${\bf y = L^T(x-z)}$ which can be solved for ${\bf x}$: ${\bf x = (L^T)^{-1}y + z}$.
%Since ${\bf K = LL}^T$, we can write ${\bf K}^{-1} = ({\bf L}^T)^{-1} {\bf L}^{-1}$ and therefore ${\bf (L^T)^{-1} = K^{-1}L}$
\end{proof}
Note that the {\elevenit inverse of the covariance matrix}, ${\bf K}^{-1}$, appears explicitly in the joint probability function 
$$f_{X_1, ..., X_n}(x_1, ..., x_n) = {1\over(2\pi)^{n/2}} {1\over|K|^{1/2}} \exp\big\{ - {1\over2} ({\bf x} - {\bf m})^T {\bf K}^{-1} ({\bf x} - {\bf m} ) \big\}$$ whereas the covariance matrix ${\bf K}$ appears in the m.g.f. 
$$\psi_{X_1, \hdots, X_n}(u_1, \hdots, u_n) = \exp\{{\bf u}^T {\bf m} + {1\over2} {\bf u}^T {\bf K} {\bf u} \}.$$

\subsection{Probabilistic Inequalities}
\label{sec:inequalities}
%Markov's inequality\\
\begin{proposition}{Markov's inequality:}
If $X$ is a random variable that takes only non-negative values, then for any value $a>0$ 

$$P\{x\ge a\} \le {E[X]\over a}$$
\end{proposition}
\begin{proof}
We give a proof for the case where $X$ is continuous with density $f(x)$.

\begin{eqnarray*}
E[X] &=& \int_0^\infty\, x f(x)\, dx \\
&=& \int_0^a\, x f(x)\, dx + \int_a^\infty\, x f(x)\, dx\\
&\ge& \int_a^\infty\, x f(x)\, dx \\
&\ge& \int_a^\infty a f(x) \, dx \\
&=& a\int_a^\infty\, f(x)\,dx\\
&=& a P\{ X \ge a\}\qed
\end{eqnarray*}

\end{proof}

%Chebyshev's inequality \\

\begin{proposition}{Chebyshev's inequality:}
If X is a random variable with finite mean $\mu$ and variance $\sigma^2$, then for any value $k>0$
$$P\{\left|X-\mu\right| \ge k \}  \le {\sigma^2\over k^2}$$
\end{proposition}

\begin{proof}
Since $(X-\mu)^2$ is a non-negative random variable, we can apply Markov's in equality (with $a=k^2$) to obtain 
$$P\big\{ (X-\mu)^2 \ge k^2 \big\} \le {E[(X-\mu)^2]\over k^2}$$ But since $(X-\mu)^2 \ge k^2$ if and only if $|X-\mu|\ge k$, the above equation is equivalent to 
$$P\big\{|X-\mu| \ge k \big\} \le {E[(X-\mu)^2]\over k^2} = {\sigma^2\over k^2} \qed$$
\end{proof}

%Kolmogorov's inequality \\
\begin{proposition}{Kolmogorov's inequality:}
Let $X_1, X_2, \hdots X_n$ be $n$ independent random variables such that $E[X_i] = 0$ and $\hbox{Var}[X_i] = \sigma_i^2 < \infty, i = 1, 2,\hdots n$. Then, for all $a>0$,
$$P \Big\{ \max_{i=1,\hdots n} \left| X_1 + \hdots X_i \right| > a \Big\}  \le \sum_{i=1}^n {\sigma_i^2\over a^2}$$
\end{proposition}

\begin{proof}
Consider all the sums $S_l = \sum_{i=1}^l X_i, 1\le l \le n$. Then $E[S_l] = 0$ because $E[X_i] = 0$.  Define an event $T$ as follows
$$ T = \Big\{ 
\begin{array}{l}
l \quad \hbox{ if $l$ is the first time}~ |S_l| > a, 1\le l \le n\\
\infty \quad \hbox{if}~ |S_l| \le a, \hbox{~for all~} 1\le l\le  n
\end{array}$$
Notice that the sets of events $\{ T= i \}, i=1,2,\hdots n$ are mutually exclusive since $\{T=l \} \cap \{ T=k\} =\emptyset$ for $1\le l \neq k \le n$. 
Also, note that $\{ T < \infty \} = \bigcup\limits_{l=1}^n \{T = l\}$.\\

The variance of $S_n$ can be written as

\begin{eqnarray*}
E[S_n^2] &=& E[(S_k + (S_n - S_k))^2]\\
&\ge& E[S_n^2; T< \infty] = \sum_{k=1}^nE[S_n^2; T=k] \\
&=& E\Big[ S_k^2 + (S_n-S_k)^2 + 2S_k(S_n-S_k); T=k\Big]
\end{eqnarray*}
$$E[S_n^2] \ge \sum_{k=1}^n E[S_k^2; T=k] + \sum_{k=1}^nE[(S_n-S_k)^2); T = k] + 2\sum_{k=1}^n E[(S_n-S_k) (S_k;T=k)]$$
Now $$\sum_{k=1}^nE[(S_n-S_k)^2; T = k] \ge 0$$ and 
$$\sum_{k=1}^n E[(S_n-S_k) (S_k;T=k)] = \sum_{k=1}^n E[(S_n-S_k)]E[(S_k;T=k)]= 0$$ by independence because $E[(S_n-S_k)] = 0$. Therefore we have found
$$E[S_n^2] \ge \sum_{k=1}^n E[S_k^2; T=k] \ge \sum_{k=1}^n (a^2) P\{ T=k\}$$ or 
$$E[S_n^2] \ge a^2 \sum_{k=1}^n P\{ T=k\} = a^2 P\{ T < \infty\}$$ which is the same as

$$P\{T<\infty\} \le {E[S_n^2]\over a^2} = \sum_{i=1}^n {\sigma_i^2\over a^2} \qed$$ 
\end{proof}

Kolmogorov's inequality can be regarded as a generalization of Chebyshev's inequality. However, 
Kolmogorov's inequality is much stronger than Chebyshev's inequality. Take $X_1, \hdots, X_2$ to be independent zero-mean random variables with variance 
$\hbox{Var}[X_i] = \sigma^2_i$, then Chebyshev's inequality gives

$$P\{|X_1 + \hdots + X_n | > a  \} \le \sum_{i=1}^n {\sigma^2_i\over a^2}$$whereas Kolmogorov's inequality gives the same bound for the probability of a larger set, namely
$$\bigcup\limits_{i=1}^n \{ |X_1 + \hdots + X_i | > a\}  $$
 
\subsection{Kronecker's Lemma}
\begin{proposition}(Kronecker's Lemma) If $a_1, a_2, \hdots $ are real numbers such that $\sum_{i=1}^\infty a_i/i < \infty$, then 
$$\lim_{\ninfty} \sum_{i=1}^n {a_i\over n} = 0.$$
\end{proposition}

\subsection{Strong Law of Large Numbers for Independent Random Variables}
\begin{theorem}
Let $X_1, X_2, \hdots $ be independent zero-mean random variables with variance $\hbox{Var}[X_i] = \sigma^2_i < \infty$. If $\sum_{i=1}^\infty \sigma^2_i/ i^2 < \infty$, then with probability 1, 
$$ {X_1 + \hdots + X_n \over n} \to 0, \quad \hbox{as~}  \ninfty$$
\end{theorem}
\begin{proof}
For any $n$ and any $a > 0$, it follows from Kolmogorov's inequality that 
\begin{equation} P\Big\{ \max_{j=1, \hdots n} \left| \sum_{i=1}^j {X_i\over i} \right| > a  \Big\} \le {\sum_{i=1}^n \hbox{Var}[X_i/i]\over a^2} \le {\sum_{i=1}^\infty \sigma^2_i/i^2 \over a^2}
\label{eqn:strongLaw}
\end{equation}
Define $$E_n = \Big\{ \max_{j=1, \hdots n} \left| \sum_{i=1}^j {X_i\over i} \right| > a  \Big\} $$
Then, because the $E_n$ is an increasing sequence of events, it follows from the continuity property of probability that
$$\lim_{\ninfty} P(E_n) = P(\lim_{\ninfty} E_n) = P(\bigcup\limits_1^\infty E_n) = P\Big\{ \max_{j\ge1} \left| \sum_{i=1}^j {X_i\over i} \right| > a   \Big\}$$
Hence, Eq. [\ref{eqn:strongLaw}] gives us 
\be P\Big\{ \max_{j\ge 1} \left| \sum_{i=1}^j {X_i\over i} \right| > a  \Big\}  \le  {\sum_{i=1}^\infty \sigma^2_i/i^2 \over a^2} \ee
or \be P\Big\{ \max_{j\ge 1} \left| \sum_{i=1}^j {X_i\over i} \right| \le a  \Big\}  \ge  1 - {\sum_{i=1}^\infty \sigma^2_i/i^2 \over a^2} \ee
Because $\max_{j\ge1} \left| \sum{_i=1}^j  X_i/i  \right| \le a$ implies that $\sum_{i=1}^\infty X_i/i \le a \le infty$ we find
\be P\Big\{ \sum_{i=1}^\infty {X_i\over i}  < \infty  \Big\} \ge  1 - {\sum_{i=1}^\infty \sigma^2_i/i^2 \over a^2} \ee
By taking $a\to \infty$ we find
 \be P\Big\{ \sum_{i=1}^\infty {X_i\over i}  < \infty  \Big\}  =  1 \ee and by Kronecker's lemma
 \be \lim_{\ninfty} P\Big\{ \sum_{i=1}^n {X_i\over i}  = 0  \Big\}  =  1 \qed \ee
 \end{proof}
 
 If the random variables are assumed not only to be independent but also identically distributed with mean $\mu$ and finite variance $\sigma^2$, then as $\sum{i=1}^\infty \sigma^2/i^2 < \infty$ by assumption, it also follows that wp1, 
 $$\lim_{\ninfty} \sum_{i=1}^n {(X_i - \mu)\over n} = 0$$ of that 
 $$\lim_{\ninfty} \sum_{i=1}^n {X_i \over n} = \mu.$$

\subsection{Central Limit Theorem}
\label{sec:CentralLimit}
%Central Limit Theorem
We will follow the method in Ross\footnote{Sheldon Ross, ``A First Course in Probability'', 9th Edition, Pearson Education, Inc., pp. 370-371 (2014).}.
\begin{theorem}
\label{thm:CentralLimit}
Let $X_1, X_2, \hdots$ be a sequence of independent and identically distributed random variables, each having mean $\mu$ and variance $\sigma^2$. The
distribution of 
$${X_1 + X_2 + ... + X_n  - n\mu\over\sigma\sqrt{n}}$$ tends to the standard normal as $\ninfty$. For $-\infty < a < \infty$,
$$ P\Bigg\{ {X_1 + X_2 +\hdots + X_n - n\mu \over \sigma\sqrt{n} } \le a \Bigg\} \rightarrow {1\over \sqrt{2\pi}} \int_{-\infty}^a \exp\left(-{x^2 / 2}\right)\,dx.$$
\end{theorem}
We will use the following lemma, which is stated without proof:\\
\begin{lemma}
\label{lem:Generating}
Let $Z_1, Z_2, \hdots$ be a sequence of random variables having distribution functions $F_{Z_n}$ and m.f.g $M_{Z_n}(t) , n\ge 1$, and let $Z$ be a random variable having distribution function $F_Z$ and m.f.g $M_Z(t)$. If 
$M_{Z_n}(t) \to M_Z(t)$ for all $t$, then $F_{Z_n}(z) \to F_Z(z)$ for all $z$ at which $F_Z(z)$ is continuous. 
\end{lemma}
In the case where $F_Z(z)$ is the standard normal random variable, then $M_Z(t) = e^{t^2/2}$. It follows from Lemma \ref{lem:Generating}\ that if 
$M_{Z_n}(t) \to e^{t^2/2}$ as $\ninfty$, then $F_{Z_n}(z) \to \Phi(z)$ as $\ninfty$ ($\Phi(z)$ is the CDF for the standard normal distribution defined in 
Eq. [\ref{eqn:Phi}]).
\begin{proof} (Proof of Theorem \ref{thm:CentralLimit}) Let us first assume that $\mu=0$ and $\sigma = 1$. We also assume that the m.g.f. of the $X_i, M(t)$, exists and is finite. 
The m.f.g for $X_i/\sqrt{n}$ is given by 
$$E\Bigg[\exp\Big\{{tX_i\over\sqrt{n}} \Big\} \Bigg] = M\Big({t\over\sqrt{n}} \Big).$$ 
Then the m.g.f. of $\displaystyle{\sum_{i=1}^n {X_i\over\sqrt{n}}}$ is given by  $\displaystyle{\Bigg[ M\Big({t\over\sqrt{n}} \Big)\Bigg]^n.}$ 
Now let $L(t) = \log M(t)$ and notice that $M(0) = 1$ and 
\bearray
L(0) &=& 0 \\
L'(0) &=& {M'(0)\over M(0)}\\
&=& \mu = 0 \\
L''(0) &=& {M(0) M''(0) - [M'(0)]^2 \over [M(0)]^2}\\
&=& E[X^2] = 1
\eearray 
In order to prove Theorem \ref{thm:CentralLimit}\/ we must show that $[M(t/sqrt{n}) \to e^{t^2/2}$ as $\ninfty$, or that 
$nL(t/\sqrt{n}) \to t^2/2$ as $\ninfty$. In order to show this, notice that 
\bearray
\lim_{\ninfty} {L(t/\sqrt{n})\over n^{-1}} &=& \lim_{\ninfty} { -L'(t/\sqrt{n}) n^{-3/2}t \over -2n^{-2}} \quad \hbox{by L'H\^opital's rule} \\
&=& \lim_{\ninfty} { -L'(t/\sqrt{n}) t \over -2n^{-1/2}} \\
&=& \lim_{\ninfty} { -L''(t/\sqrt{n}) n^{-3/2} t^2 \over -2n^{-3/2}}  \quad\hbox{also by L'H\^opital's rule} \\
&=& \lim_{\ninfty} { L''\big({t\over\sqrt{n}}\big) t^2 \over 2}  \\ 
&=&  {t^2\over2}
\eearray
Therefore the Central Limit Theorem is proven in the case $\mu =0$ and $\sigma=1$. The result can be extended to the more 
general case of independent identically distributed random variables with non-zero mean and arbitrary variance 
by transforming $X_i$ to the standard normal variable $X_i^* = (X_i - \mu)/\sigma$ and applying the result obtained above. Since
$E[X^*_i] = 0, \hbox{Var}[X^*_i] = 1$ we have directly  
\be {X^*_1 + X^*_2 +\hdots + X^*_n \over \sqrt{n} } =  {X_1 + X_2 +\hdots + X_n - n\mu \over \sigma\sqrt{n} } \label{eqn:ExtendedCentral}\ee
Since the theorem was proven for
the expression on the LHS of Eq. [\ref{eqn:ExtendedCentral}], it will also hold for the RHS since these expressions are equivalent. \qed\end{proof}

\subsection{Borel-Cantelli Lemma}
\label{sec:BorelCantelli}
%Borel-Cantelli

\begin{lemma}{Borel-Cantelli}\\

Let $A_1, A_2, \hdots $ be random events in a probability space $\Omega$.\\

1) If $\sum_{n=1}^\infty P(A_n) < \infty$, then $P(A_n, \hbox{i.o.}) = 0;$\\

2) If $A_1, A_2, \hdots$ are independent, and $\sum_{n=1}^\infty P(A_n) = \infty$, then $P(A_n, \hbox{i.o.}) = 1$\\

\end{lemma}

\begin{proof}{(Borel-Cantelli {\cal 1})}
Let $B_k$ be the event $\bigcup\limits_{i=k}^\infty A_i$ for $k=1,2,\hdots$, If $x$ is in the event $\{ A_i, \hbox{i.o.} \}$, then $x\in B_k$ for all $k$, therefore $x\in \bigcup\limits_{k=1}^\infty B_k$.\\

Conversely, if $x\in B_k$ for all $k$, then we can show that $x$ is in $\{A_i, \hbox{i.o.}\}$. This is true because $x\in B_1 = \bigcup\limits_{i=1}^\infty A_i$ means that $x\in A_{j_1}$ for some $j_1$. However, 
$x\in B_{j_1 + 1}$ implies that $x\in A_{j_1}$ for some $j_2$ that is strictly larger than $j_1$. Therefore we can produce an infinite sequence of integers $j_1 < j_2 < j_3 < \hdots$ such that $x\in A_{j_i}$ for 
all $i$. Let $E$ be the event $\{ x: x\in A_i, \hbox{i.o.} \}$. By the definition of $\limsup_{\ninfty} A_n$ we have 
$$E = \bigcap\limits_{n=1}^\infty \, \bigcup\limits_{i=n}^\infty \, A_i $$ From $E\subset B_k$ for all $k$, it follows that $P(E) \le P(B_k)$ for all $k$. By the property of union bound, we have that $P(B_k) \le \sum_{i=k}^\infty P(A_i)$. By hypothesis since $\sum_{i=1}^\infty P(A_i)$ is finite and hence $P(B_k)\rightarrow 0$ as $k\rightarrow\infty$. Therefore $P(E) = 0$.\qed
\end{proof}

\begin{proof}{(Borel-Cantelli {\cal 2})}
Let $E$ denote the set of samples that are in $A_i$ infinitely often. We have to show that complement of $E$ (denoted by $E^c$) has probability zero.\\

Taking the complement of $E$ we find using the definition in the proof of Borel-Cantelli 1) and DeMorgan's laws
$$E^c = \bigcup\limits_{k=1}^\infty \, \bigcap\limits_{i=k}^\infty \, A_i^c $$ But for each $k$, assuming that the $A_i$'s are independent, 

\begin{eqnarray*}
P\big(\bigcup\limits_{i=k}^\infty A_i^c \big) &=& \prod_{i=k}^\infty P(A_i^c) \\
&=& \prod_{i=k}^\infty (1 - P(A_i))
\end{eqnarray*}

The inequality $1 - a \le e^{-a}$ and the assumption that the sum of $P(A_i)$ diverges together imply that
$$P\big( \bigcup\limits_{i=k}^\infty A_i^c \big) \le \exp\Big( -\sum_{i=k}^\infty P(A_i) \Big)  = 0$$
Therefore $E^c$ is a union of countable number of events, each of them has probability zero. So $P(E^c) = 0$.\qed
\end{proof}
 
\subsection{Partitioned Matrices}
\label{sec:partitioned}
%Partitioned Matrices
Consider the matrix given by 
$$M = \left[
\begin{array}{c c}
 A & B \\
C & D \\
\end{array}\right],$$
where $A$ and $D$ are square matrices not necessarily of the same size and $B$ and $C$ are not necessarily square matrices. Assume $A$ and $D$ have inverses $A^{-1}$ and $D^{-1}$ respectively. Multiply the top row by $CA^{-1}$ and subtract that result from the bottom row. The resulting matrix is 
$$M' = \left[ 
\begin{array}{cc}
A & B \\
0 & D - CA^{-1}B \\
\end{array}
\right]$$ Now this type of transformation can be thought of as a particular type of elementary row operation that leaves the determinant of $M$ unchanged: $\det M = \det M'$. Using the notation $|M|$ as a shorthand for $\det M$ we can conclude $$|M| = |A| \cdot | D - CA^{-1}B|.$$ By doing a similar row operation by multiplying the second row by $BD^{-1}$ and subtracting it from the top row we obtain
$$M'' = \begin{bmatrix} 
A - BD^{-1}C & 0 \\
C & D\\
\end{bmatrix}$$ So we also have that $\det M = \det M''$ and hence $$|M| = |D| \cdot |A - BD^{-1}C|.$$ The above identities are handy in dealing with some of the applications of stochastic processes to the problem of filtering. \\

Let us develop a couple of interesting matrix identities which occur in filters. We will develop these identities by performing the inversion procedure on the matrix $M$ using the following work matrix to keep track of the inversion steps:
$$\left[
\begin{array}{c c | cc}
 A & B & I  & 0 \\
C & D & 0  & I \\
\end{array}\right],$$\\

First subtract $CA^{-1}$ times the first row from the second row and multiply the first row by $A^{-1}$. The work matrix becomes:
$$\left[
\begin{array}{c c | cc}
 I & A^{-1} B & A^{-1}  & 0 \\
0 & D - CA^{-1}B  & -CA^{-1}  & 1 \\
\end{array}\right],$$\\

Next subtract $A^{-1}B[D-CA^{-1}B]^{-1}$ times the second row from the first row and multiply the second row by $[D-CA^{-1}B]^{-1}$. The work matrix now becomes:
$$\left[
\begin{array}{c c | ll}
 I & 0 & A^{-1} - A^{-1}B[D-CA^{-1}B]^{-1}CA^{-1}  & -A^{-1}B[D-CA^{-1}B]^{-1} \\
0 & I   & -[D-CA^{-1}B]^{-1}CA^{-1}  & [D-CA^{-1}B]^{-1} \\
\end{array}\right],$$\\

Now, in order to discover the interesting identities let use an alternative ordering of row operations. In this second case we 
first subtract $BD^{-1}$ times the second row from the first row and multiply the second row by $D^{-1}$. The work matrix becomes:
$$\left[
\begin{array}{c c | cc}
 A - BD^{-1}C  & 0  & I  & -BD^{-1} \\
D^{-1}C & I  &  0  & D^{-1} \\
\end{array}\right],$$\\

Next subtract $D^{-1}C[A-BD^{-1}C]^{-1}$ times the first row from the second row and multiply the first row by $[A-BD^{-1}C]^{-1}$. The work matrix now becomes:
$$\left[
\begin{array}{c c | ll}
 I & 0 & [A-BD^{-1}C]^{-1}  & -[A-BD^{-1}C]^{-1}BD^{-1} \\
0 & I   & -D^{-1}C[A-BD^{-1}C]^{-1}  & D^{-1}  -D^{-1}C[A-BD^{-1}C]^{-1}BD{-1}  \\
\end{array}\right],$$

Since these are two ways of obtaining the inverse of $M$, the matrix elements must be equal term-by-term. Hence we obtain some interesting identities. For example: 
using Partitioned Matrices we have found the Sherman-Morrison-Woodbury formula
\be [A -BD^{-1}C]^{-1} = A^{-1} - A^{-1}B[D-CA^{-1}B]^{-1}CA^{-1}. \label{eqn:InverseMatrixIdentity}\ee 

%%%%%%%
Let us apply the ideas of joint random variables, conditional probabillity and expectation to a random $(n+m)$-vector $Z^T = [X^T,Y^T]$, where $X$ is an $n$-vector and $Y$ is an $m$-vector and assume that $Z$ is a multivariate Gaussian $Z \sim N(m_Z, P_{ZZ})$ (with $m_Z = E[Z]$ is a vector of means and $P_{ZZ} = E[(z-m_Z)(z-m_Z)^T]$ is the covariance matrix). This also assumes that $X$ and $Y$ are jointly random Gaussian variables. 
Then $Z$ has a probability density function given by 
\begin{equation} f_Z(z) = {1\over \sqrt{(2\pi)^{n+m} |P_{ZZ}|} } \exp\big\{-{1\over2} (z-m_z)^T P_{ZZ}^{-1} (z-m_z) \big\}.\label{eqn:multiGaussian}\end{equation}
Now we can use the partition matrix approach given in Section [\ref{sec:partitioned}] to write the covariance matrix $P_{ZZ}$ as
$$P_{ZZ} = E[(z - m_Z)(z-m_Z)^T] = 
\begin{bmatrix}
E[(X-m_X)(X-m_X)^T]  & E[(X-m_X)(Y-m_Y)^T] \\
E[(Y-m_Y)(X-m_Z)^T]  & E[(Y-m_Y)(Y-m_Y)^Y]
\end{bmatrix}
=\begin{bmatrix}
P_{XX}  & P_{XY} \\
P_{YX} & P_{YY} 
\end{bmatrix}
$$

Then we have 
\begin{theorem}
\label{thm:jointlyDistributed}
If the random vectors $X$ and $Y$ are distributed according to Eq. [\ref{eqn:multiGaussian}]\, then $X$ and/or $Y$ is also a multivariate Gaussian (with appropriate parameters). 
\end{theorem}
\begin{proof} Let show the theorem holds for $Y$. Accordingly let use consider a transformation of variables to $W^T = [W_1^T, W_2^T]$ such that 
$$W_1 = X - P_{XY}P_{YY}^{-1} Y, \quad W_2 = Y.$$
Observe that the Jacobian of the above transformation is $1$.\\

From Theorem [\ref{thm:Gaussian}] $W$ is normally distributed. Let us compute $E[(W_1 - E[W_1])(W_2 - E[W_2])]$,
\begin{eqnarray*}
E[W_1] &=& m_X - P_{XY}P_{YY}^{-1}m_Y\\
W_1 - E[W_1] &=& (X-m_X) - P_{XY}P_{YY}^{-1}(Y-m_Y) \\
E[W_2] &=& m_Y\\
W_2 - E[W_2] &=& (Y-m_Y) \\
\end{eqnarray*} and so
\begin{eqnarray*}
E[(W_1 -E[W_1])(W_2 - E[W_2])^T] &=& E [ \big( (X-m_X) - P_{XY}P_{YY}^{-1}(Y-m_Y)\big) \big(Y -m_Y\big)^T] \\
&=& E[(X-m_X)(Y-m_Y)^T] - E[P_{XY}P_{YY}^{-1}(Y-m_Y)(Y-m_Y)^T]\\
&=& P_{XY} - P_{XY}P_{YY}^{-1}P_{YY} \\
&=& P_{XY} - P_{XY} = 0
\end{eqnarray*} and by symmetry $E[(W_2 -E[W_2])(W_1 - E[W_1])^T] = 0$.\\

Also, 
\begin{eqnarray*}
E[(W_1 -E[W_1])(W_1 - E[W_1])^T] &=& E [ \big( (X-m_X) - P_{XY}P_{YY}^{-1}(Y-m_Y)\big) \big( (X-m_X) - P_{XY}P_{YY}^{-1}(Y-m_Y)^T \big)] \\
&=& E[(X-m_X)(X-m_X)^T] - E[P_{XY} P_{YY}^{-1} (Y-m_Y)(Y-m_Y)^T P_{YY}{-1}^T P_{XY}^T]\\
&=& P_{XX} - P_{XY}P_{YY}^{-1}P_{YX}
\end{eqnarray*} 

Therefore
$$P_{ZZ} = E[(z - m_Z)(z-m_Z)^T] =  
\begin{bmatrix}
P_{XX} -P_{XY}P_{YY}^{-1}P_{XY}  & 0 \\
0 & P_{YY}
\end{bmatrix}\qed$$\end{proof}
Therefore we can say that $X$ and $Y$ are marginally Gaussian (with appropriate parameters).\\

It is clear that if two vectors of random variables $X$ and $Y$  are independent, then they are uncorrelated.  An important property of vectors of Gaussian random variables is the converse.
\begin{theorem}If two jointly distributed Gaussian random vectors are uncorrelated then they are also independent. \end{theorem}
\begin{proof}
Let $X$ and $Y$ are random vectors of dimension $n$ and $m$ respectively, that are jointly Gaussian distributed and uncorrelated
Define $Z$ to be a random vector of dimension $(n+m)$ composed of the components of $X$ and $Y$ as above:
$$Z^T = [X^T,Y^T], \quad Z = \left[
\begin{array}{c}
X \\
\hline
Y \\
\end{array}\right]$$
Then $Z$ has a mean given by 
$$m_Z = E[Z] 
= \left[\begin{array}{c}
E[X] \\
\hline
E[Y] \\
\end{array}\right]
= \left[\begin{array}{c}
m_X \\
\hline
m_Y\\
\end{array}\right]
$$
Because $X$ and $Y$ are uncorrelated, then we have 
$$E[XY^T] = m_Xm_Y^T, \quad E[YX^T] = m_Ym_X^T$$ and the covariance $P_{ZZ}$ becomes

The second moment of $Z$ is given by 
$$E[ZZ^T] 
= \left[\begin{array}{c|c}
E[XX^T] & E[XY^T] \\  
\hline
E[YX^T] & E[YY^T]\\
\end{array}\right], \quad P_{ZZ} = E[ZZ^T] -m_Zm_Z^T  $$
Therefore the covariance $P_{ZZ}$ becomes block diagonal:
$$P_{ZZ} = E[ZZ^T] - m_Zm_Z^T 
= \left[\begin{array}{c|c}
E[XX^T] & m_Xm_Y^T \\  
\hline
m_Ym_X^T & E[YY^T]\\
\end{array}\right] 
- \left[\begin{array}{c|c}
m_Xm_X^T & m_Xm_Y^T \\  
\hline
m_Ym_X^T & m_Ym_Y^T\\
\end{array}\right] 
= \left[\begin{array}{c|c}
P_{XX} & 0 \\  
\hline
0 & P_{YY}\\
\end{array}\right] 
 $$
 Therefore the determinant of $P_{ZZ}$ is given by the product
 $$|P_{ZZ}| = |P_{XX}| |P_{YY}|$$
Let $\xi$ be the vector composed of $n$ $x$-values and $m$ $y$-values 
$$\xi
= \left[\begin{array}{c}
x \\
\hline
y \\
\end{array}\right]$$
Given the block diagonal structure of $P_{ZZ}$ we can form the quadratic form 
$[\xi - m_Z]^T P_{ZZ}^{-1} [\xi - m_Z]$ which can be expanded using the vector representation of $\xi$ in terms of $x$ and $y$ values

\begin{eqnarray*}
[{\bf \xi} - {\bf m_Z}]^T {\bf P^{-1}_{ZZ}}[{\bf \xi} - {\bf m_Z}]  &=& 
\left[\begin{array}{c}
x - m_X \\
\hline
y-m_Y \\
\end{array}\right]^T
\left[\begin{array}{c|c}
P_{XX}^{-1} & 0 \\  
\hline
0 & P_{YY}^{-1}\\
\end{array}\right] 
\left[\begin{array}{c}
x - m_X \\
\hline
y-m_Y \\
\end{array}\right]\\ 
&=&
[x - m_X]^T P_{XX}^{-1} [x - m_X] + [y-m_Y]^TP_{YY}^{-1}[y-m_Y]
\end{eqnarray*}
Therefore the joint probability distribution can be written as 

\bearray f_Z(\xi)  &=& {1\over(2\pi)^{(n+m)/2}} {1\over|P_{ZZ}|^{1/2}} \exp\big\{ - {1\over2} ({\bf \xi} - {\bf m_Z})^T {\bf P^{-1}_{ZZ}} ({\bf \xi} -{\bf m_Z})\big\} \\
&=& \left[{1\over(2\pi)^{n/2}} {1\over|P_{XX}|^{1/2}} \exp\big\{ - {1\over2} ({\bf x} - {\bf m_X})^T {\bf P^{-1}_{XX}}  ({\bf x} -{\bf m_X})\big\}\right] \\ 
&\times& \left[{1\over(2\pi)^{m/2}} {1\over|P_{YY}|^{1/2}} \exp\big\{ - {1\over2} ({\bf y} - {\bf m_Y})^T {\bf P^{-1}_{YY}} ({\bf y} -{\bf m_Y})\big\}\right]
\eearray
And so if the Gaussian random vectors $X$ and $Y$ are uncorrelated, then their joint probability density factors into 
\be f_Z(\xi) = f_X(x) \times f_Y(y) \ee and therefore $X$ and $Y$ are also independent random variables.\qed\end{proof}
By extension, if $X_1, X_2, \hdots, X_n$ are normally distributed random variables and are pair-wise uncorrelated, then they are independent. 
%Also, if the Gaussian random vectors $X$ and $Y$ are uncorrelated, then they are independent. 


\begin{theorem} Let $X$ and $Y$ be jointly Gaussian distributed as in Eq. [\ref{eqn:multiGaussian}]. Then the conditional probability density of $X$ given $Y$ is normal with mean 
\be m_X + P_{XY}P_{YY}^{-1}(Y-m_Y) \label{eqn:ConditionalMean}\ee and covariance matrix \be P_{XX} - P_{XY}P_{YY}^{-1}P_{YX}.\label{eqn:ConditionalVariance}\ee 
\end{theorem}
\begin{proof}
We have from Theorem [\ref{thm:jointlyDistributed}] that $W_1$ and $W_2$ are Gaussian and independent. Therefore, their joint density, which is also Gaussian, is the product of their individual probability distributions: 
\begin{eqnarray*}
W_1 &\sim& N(m_X - P_{XY}P_{YY}^{-1} m_Y , ~~P_{XX} - P_{XY}P_{YY}^{-1}P_{YX}) \\
W_2  &\sim& N(m_Y, P_{YY})
\end{eqnarray*}

Therefore the combined joint probability distribution for $W$ can be written as 

\begin{eqnarray*}
f_{W_1,W_2}(w_1,w_2) &=& {1\over\sqrt{2\pi)^n |P_{XX} - P_{XY}P_{YY}^{-1}P_{YX} | }} \\ 
&\null& \exp 
\big\{ -{1\over2}(w_1 - m_X + P_{XY}P_{YY}^{-1}m_Y)^T(P_{XX} - P_{XY}P_{YY}^{-1}P_{YX})^{-1} \\
&\null&  (w_1 - m_X + P_{XY}P_{YY}^{-1}m_Y) \big\} \\
&\null&  {1\over\sqrt{2\pi)^n |P_{YY}|} } \exp\big\{-{1\over2}(w_2 - m_Y)^TP_{YY}^{-1}(w_2 - m_Y)  \big\}
\end{eqnarray*}

Since the Jacobian of the transformation between $X, Y$ and $W_1, W_2$ is equal to 1, we can recover the density $f_{X,Y}(x,y)$ by eliminating $W_1$ and $W_2$ in terms of $X$ and $Y$. Doing do, we obtain
\begin{eqnarray*}
f_{X,Y}(x, y) &=& {1\over\sqrt{2\pi)^n |P_{XX} - P_{XY}P_{YY}^{-1}P_{YX} | }} \\ 
&\null& \exp 
\big\{ -{1\over2}[x - m_X  - P_{XY}P_{YY}^{-1}(y - m_Y)]^T(P_{XX} - P_{XY}P_{YY}^{-1}P_{YX})^{-1} \\
&\null&  [x - m_X -  P_{XY}P_{YY}^{-1}(y - m_Y)] \big\} \\
&\null&  {1\over\sqrt{2\pi)^n |P_{YY}|} } \exp\big\{-{1\over2}(y - m_Y)^TP_{YY}^{-1}(y - m_Y)  \big\}
\end{eqnarray*}

Now the right hand side of the above is the marginal probability function of $Y$. So by comparison with Eq. [\ref{eqn:jointConditional}] we see that the conditional density function of $X$ given $Y$ is

\begin{equation}
\label{eqn:conditionalGaussian}
\begin{array}{cl}
f_{X|Y}(x|y) &= {1\over\sqrt{2\pi)^n |P_{XX} - P_{XY}P_{YY}^{-1}P_{YX} | }} \\ \\
\null & \exp \big\{ -{1\over2}[x - m_X  - P_{XY}P_{YY}^{-1}(y - m_Y)]^T [P_{XX} - P_{XY}P_{YY}^{-1}P_{YX} ]^{-1} \\ \\
\null & [x - m_X -  P_{XY} P_{YY}^{-1} (y - m_Y) ] \big\} 
\end{array}
\end{equation}

Eq. [\ref{eqn:conditionalGaussian}] is important for applications in Filtering theory. 

\end{proof}

%\section{Glossary of Terms}
%Ergodic\\
%1) systems such that the {\elevenit time average}\/ equals the {\elevenit assemble average} are called {\elevenit ergodic}. \\
%2). Markov chains: positive, recurrent, aperiodic states are called {\elevenit ergodic}.\\
%3). Markov chains: if the limiting probability for finding the system in state $j$ is independent of the initial state $i$, then the state $j$ is called {\elevenit ergodic}.
%%%%%%%


\section{Acknowledgements}

These notes grew out of discussions I have had with Professor Howard Weiner of the University of California, Davis. Many of the methods are taken directly from my notes of Professor Weiner's explanations. 
\clearpage

\begin{thebibliography}{9}
 \bibitem{Protter}{M.H. Protter and C.B. Morrey, ``A First Course in Real Analysis'', Springer-Verlag, Inc., New York (1977).}
 \bibitem{Hall}{F. M. Hall, ``An Introduction to Abstract Algebra'', Volume 2, Cambridge University Press, (1969).}
 \bibitem{Stensby}{John Stensby, ``EE603 Class Notes'', \url{http://www.ece.uah.edu/courses/ee385/}, University of Alabama, Huntsville (2016).}
 \bibitem{Baudoin}{Fabrice Baudoin,``Stochastic Processes and Brownian Motion'',\url{http://www.math.purdue.edu/~fbaudoin/MA539.pdf}, Purdue University, Indiana (2010).}
 \bibitem{vanZanten}{Harry van Zanten,``An Introduction to Stochastic Processes in Continuous Time'', \url{http://www.math.vu.nl/sto/onderwijs/sp/sp_2007.pdf}, University of Leiden (2007).}
 \bibitem{Weiner1}{Howard Weiner, UC Davis Mathematics Department, private communication (2007-2017).}
 \bibitem{Mahalanobis}{Mahalanobis, Prasanta Chandra (1936).``On the generalised distance in statistics'', Proceedings of the National Institute of Sciences of India. 2 (1): 49-55.}
 \bibitem{Einstein}{Albert Einstein, ``Investigations on the Theory of the Brownian Movement'', Dover Publications, Inc.(1956).}
 \bibitem {Ross1} {Sheldon Ross, ``A First Course in Probability'', First Edition, Macmillan Publishers Co, Inc., New York, NY (1976).}
 \bibitem {Ross2} {Sheldon Ross, ``Introduction to Probability Models'', Sixth Edition, Academic Press, San Diego, CA  (1997).}
 \bibitem{Parzen}{Emanuel Parzen, ``Stochastic Processes'', Holden-Day, Inc., San Francisco, CA (1962).}
 \bibitem{Medhi}{Jyotiprasad Medhi, ``Stochastic Processes'', Wiley Eastern Limited, New Delhi (1983).} 
 \bibitem{Jazwinski}{Andrew H. Jazwinski,``Stochastic Processes and Filtering Theory'', Dover Publications, Inc., Mineola, NY (2007).}
 \bibitem{Wiener1}{Norbert Wiener, ``Cybernetics: or Control and Communication in the Animal and the Machine'', The MIT Press, Cambridge, MA (1948).}
 \bibitem{Wiener2}{Norbert Wiener,``Extrapolation, Interpolation, and Smoothing of Stationary Time Series with Engineering Applications'', First Paperback Edition, The MIT Press, Cambridge, MA (1964). }  
 \bibitem{Wiener3}{Norbert Wiener, ``Generalized Harmonic Analysis'', Acta Mathematica, V. 55, p. 117 (1930).}  
 \bibitem{Oskendal1}{Bernt  {\O}skendal, ``Stochastic Differential Equations, An Introduction with Applications'', Third Edition, Springer-Verlag, Berlin (1992).}
 \bibitem{Oskendal2}{Bernt  {\O}skendal, ``Stochastic Differential Equations, An Introduction with Applications'', Sixth Edition, Springer-Verlag, Berlin (2005).}
 \bibitem{Doob}{J. L. Doob, ``Stochastic Processes'', John Wiley \& Sons, Inc., New York, (1953).}
 \bibitem{Sjogren}{Lennart Sjogren. ``Brownian Motion: Langevin Equation'', Chapter 6, \url{http://physics.gu.se/~frtbm/joomla/media/mydocs/LennartSjogren/kap6.pdf}}
 \bibitem{CalTech}{``Statistical Mechanics'', Physics 127b \url{http://www.cmp.caltech.edu/~mcc/Ph127/b/Lecture16.pdf}}
 \end{thebibliography}

%\begin{description}
%\elevenitem[\elevenit Reflexive.] Since $x = e^{-1}xe$, x is conjugate to itself.
%\elevenitem[\elevenit Symmetric.] If $y=g^{-1}xg$ we have $x=gyg^{-1} = g^{-1})^{-1}y g^{-1}$.
%\elevenitem[\elevenit Transitive.] If $y=g^{-1}xg$ and $z=h^{-1}yh$ we have $z=h^{-1}g^{-1}xgh=(gh)^{-1}x(gh).$
%\end{description}

%\begin{eqnarray*}
%   P_1 & = & (E_1,0,0,-E_1),      \\
%   P_2 & = & (E_2,E_2\sin\theta\cos\phi,E_2\sin\theta\sin\phi,
%               -E_2\cos\theta),   \\
%   P_3 & = & (E_3,0,0,\beta E_3),      \\
%   q   & = & (E_1 - E_2,-E_2\sin\theta\cos\phi,-E_2\sin\theta\sin\phi,
%                    E_2\cos\theta - E_1).
%  \end{eqnarray*}

%\begin{figure*}[hb]
%\epsfysize=6cm %%%% whatever vertical size you want in cm or inches
%\centerline{\epsfbox{user$1:[smith.tex.eps]feyn.eps}}
%\caption[fig1]{Definition of Kinematic Variables.}
%\label{fig:feyn}
%\end{figure*}

%  \begin{eqnarray*}
%   y\     & \approx & 1 - {E_2\over E_1} {(1+\cos\theta)\over2}, \\
%    \qsq \ & \approx & {E_2\over E_1}{(1-\cos\theta)\over2}, \\
%    x      & \approx & {Q^2\over sy}.
%  \end{eqnarray*}

%\[ \begin{array}{lll} 
%  {\cal M}^{++}=({\cal M}^{22} + {\cal M}^{11})/2, 
%& {\cal M}^{+0}=({\cal M}^{23} -i{\cal M}^{13})/\sqrt{2},
%& {\cal M}^{+-}=({\cal M}^{22} - {\cal M}^{11})/2, \\ [8pt]
%  {\cal M}^{0+}=({\cal M}^{32} +i{\cal M}^{31})/\sqrt{2}, 
%& {\cal M}^{00}= {\cal M}^{33}, 
%& {\cal M}^{0-}=({\cal M}^{32} -i{\cal M}^{31})/\sqrt{2}, \\ [8pt]
%  {\cal M}^{-+}=({\cal M}^{22} - {\cal M}^{11})/2, 
%& {\cal M}^{-0}=({\cal M}^{23} +i{\cal M}^{13})/\sqrt{2}, 
%& {\cal M}^{--}=({\cal M}^{22} + {\cal M}^{11})/2. \\ [8pt]
%\end{array} \] 

%\[ \begin{array}{lll} 
%  {\cal M}^{++}=({\cal M}^{22} + {\cal M}^{11})/2, 
%& {\cal M}^{+0}= {\cal M}^{23}/\sqrt{2},
%& {\cal M}^{+-}=({\cal M}^{22} - {\cal M}^{11})/2, \\ [8pt]
%  {\cal M}^{0+}= {\cal M}^{32}/\sqrt{2}, 
%& {\cal M}^{00}= {\cal M}^{33}, 
%& {\cal M}^{0-}= {\cal M}^{32}/\sqrt{2}, \\ [8pt]
%  {\cal M}^{-+}=({\cal M}^{22} - {\cal M}^{11})/2, 
%& {\cal M}^{-0}= {\cal M}^{23}/\sqrt{2}, 
%& {\cal M}^{--}=({\cal M}^{22} + {\cal M}^{11})/2. 
%\end{array} \]
%
%\begin{thebibliography}{99}
%\bibitem{BCDMS} J.J.Aubert et al., EMC Collaboration,
% Nucl.Phys. B259(1985)189;\\
% A.C.Benvenuti et al., BCDMS Collaboration, Phys.Lett.B223(1989)485
%\bibitem{kaiserf}     F. Eisele, {\em First Results from the H1
%Experiment at HERA}, and
%                      F. Brasse, {\em The H1 Detector at HERA},
%                      Invited talks, Proceedings of the
%            26th International Conference on High Energy Physics,
%                   Dallas (1992) and DESY preprint 92-140 (1992)
%\bibitem{H1-249} I.\, Abt, J.R.\,Smith, {\em MC Upgrades to Study
%Untagged Events}, H1 Internal Note H1-249, (1992).
%H1-249 considered only the diagonal elements of the Photon Flux
%for photons coupled to Spin-1/2 particles.
%\end{thebibliography}
 %
 \end{document}
