%   11/11/92 211121724  MEMBER NAME  REPORT1  (TEXFILES) M  TEX
%format: latex    %hyphen: English
%\documentstyle[11pt,leqno,epsfig]{note}
%\documentstyle[11pt,epsfig]{note}
%\documentstyle[11pt,epsfig,pstimes,psmath]{note}
\documentstyle[epsf]{article}
\font\twelverm=cmr12
\font\tenbf=cmbx10
\font\tenrm=cmr10
\font\tenit=cmti10
\font\elevenbf=cmbx10 scaled\magstep 1
\font\elevenrm=cmr10 scaled\magstep 1
\font\elevenit=cmti10 scaled\magstep 1
\font\ninebf=cmbx9
\font\ninerm=cmr9
\font\nineit=cmti9
\font\eightbf=cmbx8
\font\eightrm=cmr8
\font\eightit=cmti8
\font\sevenrm=cmr7
%
%  `pstimes.sty' sets \rm to be Times, and `psmath.sty' sets the math
%  in Times as well.
%
\textwidth = 6.5truein
\textheight = 8.5truein
\oddsidemargin=0.2truein
\evensidemargin=0.2truein
\topmargin=0.50truein
%\notename{UCD/IIRPA} \notenumber{92-24}
\begin{document}
\null
%\hbox to \hsize{\hfill UCD/IIRPA~92-24}
\begin{center}{{\tenbf Derivatives and Commutators}}
		\vglue 1.0cm
 	       {\tenrm JOHN R.\,\,SMITH\\}
	        %\baselineskip=13pt
	       %{\tenit Physics Department\\
	        %University of California, Davis\\}
                 % \baselineskip=12pt
               %{\tenit Davis, CA 95616, USA\\}
\end{center}
%
%\newmathalphabet{\oldcal}
%\addtoversion{times}{\oldcal}{cmsy}{m}{n}
\newcommand{\unit}[1]{\mbox{\it #1}}
\newcommand{\subs}[1]{\mbox{\scriptsize\it #1}}
\def\Q2{$Q^2$}
\def\sgp{$\sigma_{\gamma p}\:$}
\mathchardef\Lcur="324C
\def\crfive{\cr\noalign{\vskip 5pt}}
\def\crten{\cr\noalign{\vskip 10pt}}
\def\cofac{\hbox{cofactor}}
\def\gij{g_{ij}}
\def\gijinv{g^{ij}}
\def\half{1\over2}
\def\gab{g_{\alpha\beta}}
\def\gabinv{g^{\alpha\beta}}
\def\gmunu{g_{\mu\nu}}
\def\gmunuinv{g^{\mu\nu}}
\def\gmu{\gamma^\mu}
\def\gmudag{\gamma^{\mu\dag}}
\def\gnu{\gamma^\nu}
\def\gmuc{\gamma_\mu}
\def\gnuc{\gamma_\nu}
\def\gzero{\gamma^0}
\def\gone{\gamma^1}
\def\gtwo{\gamma^2}
\def\gthree{\gamma^3}
\def\gfive{\gamma_5}
\def\gfive{\gamma_5}
\def\gzeroc{\gamma_0}
\def\gonec{\gamma_1}
\def\gtwoc{\gamma_2}
\def\gthreec{\gamma_3}
\def\qsq{{Q^2\over4E_1^2}}
\def\qr2{{Q^2\over2}}
\def\q2{{Q^2}}
\def\hy{\hat{y}}
\def\be{\begin{equation}}
\def\ee{\end{equation}}
\newcommand{\bra}[1]{\langle #1|}
\newcommand{\ket}[1]{|#1\rangle}
\newcommand{\braket}[2]{\langle #1|#2\rangle}

\newcommand{\Tr}{{\rm Tr}}

%\maketitle

%some definitions
%\setlength{\oddsidemargin}{0cm}
%\setlength{\textwidth}{17cm}
%\setlength{\topmargin}{-3cm}
%\setlength{\textheight}{25cm}
%\setlength{\unitlength}{1mm}
%\hsize=6.0 true in
%\vsize=8.40 true in
%\pagestyle{plain}
\parindent = 0.cm
\vglue 0.3cm
\rightskip=3pc
\leftskip=3pc
\twelverm

%\noindent

\section{Derivatives and Commutators}

Using the Chain Rule of Calculus we define the derivative of $X^n$ as

$${dX^n\over dt} = nX^{n-1}\, {dX\over dt}$$

We ask: Can the derivative operation be defined in terms of discrete operations, or operators which appear to be discrete?\\

Let use imagine two operators $A$ and $B$. The commutator of $A$ and $B$ is written as  $[A,B]$ and is defined as 

$$\left[A,B\right] = AB - BA$$\\

For example the operators may be defined by their action on a set of functions. Such is the case in Quantum Mechanics for the operators $p$ and $q$ which play the role of kinematic variables corresponding to ``observables'' of momentum and position.\\

Suppose the operators $A$ and $B$ commute with their commutaors, i.e., $$[B, [A,B]] = [A, [A,B]] = 0$$

Consider the following procedure: 

\begin{eqnarray*}
[A,B^n] &=& AB^n - B^nA \\
             &=& ABB^{n-1} - BAB^{n-1}  +  B(AB)B^{n-2} - B(BA)B^{n-3} + ... + B^{n-1}AB - B^{n-1}BA\\
             &=& [A,B] B^{n-1} + B[A,B]B^{n-2} +  ... + B^{n-1}[A,B]\\
\end{eqnarray*}

Using the fact that $B$ commutes with $[A,B]$ we obtain

$$[A,B^n] = B^{n-1}[A,B] + B^{n-1}[A,B] + ...+ B^{n-1}[A,B] = nB^{n-1} [A,B]$$

Also, using $[A^n,B] = - [B,A^n]$ we can also obtain $$[A^n,B] = - nA^{n-1}[B,A] = nA^{n-1}[A,B]$$\\

It seems therefore that we can get similar properties as differentiation using operators $A$ and $B$ by identifying the 
derivative operation by this correspondence: $$ {dA\over dt} = [A,B]$$ and therefore $${dA^n\over dt} = [A^n,B] = nA^{n-1} {dA\over dt} = nA^{n-1} [A,B]$$

The above correspondence appears to be fundamental to Quantum Mechanics using the dynamical variables $q = A$ and $p = B$ which lead to the Schr\"odinger picture with position operator $q = x$ and momentum operator $p = -i\hbar {\partial\over \partial_x}$ with the famous commutation rule $$\large [q,p] = i\hbar$$.

Perhaps such commutation algebra can be used to eliminate the derivative at a more fundamental level and free mechanics from continuous coordinates? \\

Also, observe that the Right Hand Side (RHS) of the commutation rule can be written in terms of the identity operator $I$ (in matrix language $I$ is the $n \times n$ unit matrix for some finite vector space of dimension $n$) as follows:

$$\large [q,p] = i\hbar\, I$$ and so the commutator of $q$ and $p$ is a multiple of the unit matrix $I$. This presents an interesting problem because there is a matrix representation associated with our operators $A$ and $B$ (or $p$ and $q$) -- this is similar to the correspondence between the Heisenberg pictures and the Schr\"odinger pictures. Therefore we can think of $A$ and $B$ as``square" matrices corresponding to a transformation on a linear vector space. But this is exactly where a complication develops. Because, for any square matrices $A$ and $B$ we have that the trace operation is cyclic, viz. $\Tr(AB) = \Tr(BA)$ and hence $$\Tr([A,B]) = \Tr(AB - BA) = \Tr(AB) - \Tr(BA) = 0.$$ This fact contradicts the commutation relation because $\Tr(I) = n$ for any finite dimensional vector space in $n$ dimensions. This contraction appears to go away if we use infinite-dimensional vector spaces! This is one way to understand why infinite dimensional vector spaces (etc. with scalar products, etc. Hilbert Spaces) are fundamental to Quantum Mechanics. \\

Conclusion -- Commutators have similar properties as derivative operators. Perhaps commutators are more useful in situations were there is no way to define a derivative due to fundamental limits on discreteness of the coorindates in space-time.


%\begin{description}
%\item[\it Reflexive.] Since $x = e^{-1}xe$, x is conjugate to itself.
%\item[\it Symmetric.] If $y=g^{-1}xg$ we have $x=gyg^{-1} = g^{-1})^{-1}y g^{-1}$.
%\item[\it Transitive.] If $y=g^{-1}xg$ and $z=h^{-1}yh$ we have $z=h^{-1}g^{-1}xgh=(gh)^{-1}x(gh).$
%\end{description}

%\begin{eqnarray*}
%   P_1 & = & (E_1,0,0,-E_1),      \\
%   P_2 & = & (E_2,E_2\sin\theta\cos\phi,E_2\sin\theta\sin\phi,
%               -E_2\cos\theta),   \\
%   P_3 & = & (E_3,0,0,\beta E_3),      \\
%   q   & = & (E_1 - E_2,-E_2\sin\theta\cos\phi,-E_2\sin\theta\sin\phi,
%                    E_2\cos\theta - E_1).
%  \end{eqnarray*}

%\begin{figure*}[hb]
%\epsfysize=6cm %%%% whatever vertical size you want in cm or inches
%\centerline{\epsfbox{user$1:[smith.tex.eps]feyn.eps}}
%\caption[fig1]{Definition of Kinematic Variables.}
%\label{fig:feyn}
%\end{figure*}

%  \begin{eqnarray*}
%   y\     & \approx & 1 - {E_2\over E_1} {(1+\cos\theta)\over2}, \\
%    \qsq \ & \approx & {E_2\over E_1}{(1-\cos\theta)\over2}, \\
%    x      & \approx & {Q^2\over sy}.
%  \end{eqnarray*}

%\[ \begin{array}{lll} 
%  {\cal M}^{++}=({\cal M}^{22} + {\cal M}^{11})/2, 
%& {\cal M}^{+0}=({\cal M}^{23} -i{\cal M}^{13})/\sqrt{2},
%& {\cal M}^{+-}=({\cal M}^{22} - {\cal M}^{11})/2, \\ [8pt]
%  {\cal M}^{0+}=({\cal M}^{32} +i{\cal M}^{31})/\sqrt{2}, 
%& {\cal M}^{00}= {\cal M}^{33}, 
%& {\cal M}^{0-}=({\cal M}^{32} -i{\cal M}^{31})/\sqrt{2}, \\ [8pt]
%  {\cal M}^{-+}=({\cal M}^{22} - {\cal M}^{11})/2, 
%& {\cal M}^{-0}=({\cal M}^{23} +i{\cal M}^{13})/\sqrt{2}, 
%& {\cal M}^{--}=({\cal M}^{22} + {\cal M}^{11})/2. \\ [8pt]
%\end{array} \] 

%\[ \begin{array}{lll} 
%  {\cal M}^{++}=({\cal M}^{22} + {\cal M}^{11})/2, 
%& {\cal M}^{+0}= {\cal M}^{23}/\sqrt{2},
%& {\cal M}^{+-}=({\cal M}^{22} - {\cal M}^{11})/2, \\ [8pt]
%  {\cal M}^{0+}= {\cal M}^{32}/\sqrt{2}, 
%& {\cal M}^{00}= {\cal M}^{33}, 
%& {\cal M}^{0-}= {\cal M}^{32}/\sqrt{2}, \\ [8pt]
%  {\cal M}^{-+}=({\cal M}^{22} - {\cal M}^{11})/2, 
%& {\cal M}^{-0}= {\cal M}^{23}/\sqrt{2}, 
%& {\cal M}^{--}=({\cal M}^{22} + {\cal M}^{11})/2. 
%\end{array} \]
%
%\begin{thebibliography}{99}
%\bibitem{BCDMS} J.J.Aubert et al., EMC Collaboration,
% Nucl.Phys. B259(1985)189;\\
% A.C.Benvenuti et al., BCDMS Collaboration, Phys.Lett.B223(1989)485
%\bibitem{kaiserf}     F. Eisele, {\em First Results from the H1
%Experiment at HERA}, and
%                      F. Brasse, {\em The H1 Detector at HERA},
%                      Invited talks, Proceedings of the
%            26th International Conference on High Energy Physics,
%                   Dallas (1992) and DESY preprint 92-140 (1992)
%\bibitem{H1-249} I.\, Abt, J.R.\,Smith, {\em MC Upgrades to Study
%Untagged Events}, H1 Internal Note H1-249, (1992).
%H1-249 considered only the diagonal elements of the Photon Flux
%for photons coupled to Spin-1/2 particles.
%\end{thebibliography}
 %
 \end{document}
